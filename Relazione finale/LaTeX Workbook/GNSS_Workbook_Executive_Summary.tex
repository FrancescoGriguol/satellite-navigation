\documentclass[a4paper,11pt,twoside]{book}

\usepackage{xcolor}
\usepackage[italian]{babel}
\usepackage{indentfirst}
\usepackage{amsmath}
\usepackage[margin=1.5cm,footskip=0.6cm]{geometry}
\usepackage{graphicx}
\graphicspath{ {figures/} }
\usepackage{array}
\newcolumntype{M}[1]{>{\centering\arraybackslash}m{#1}}
\usepackage{float}
\usepackage{dsfont}
\usepackage{amssymb}
\usepackage[bottom]{footmisc}
\usepackage{subfig}
\usepackage{wrapfig}
\usepackage{lipsum}
\usepackage{graphicx,wrapfig,lipsum}
\usepackage{amsfonts}
\usepackage[shortlabels]{enumitem}
\usepackage{amsfonts}
\usepackage{cancel}
\usepackage{bm}
\newcommand{\uvec}[1]{\boldsymbol{\hat{\textbf{#1}}}}
\usepackage{fancyhdr}
\usepackage[utf8]{inputenc}
\usepackage[font=small]{caption}
\usepackage{mathrsfs}
\usepackage{euscript}
\usepackage{eurosym}
\usepackage{titlesec}
\titleformat{\chapter}[display]
{\normalfont\bfseries}{}{0pt}{\Huge}



\begin{document}
	
	\frontmatter
	
	\begin{titlepage}
		
		\begin{figure}[H]
			\centering
			\includegraphics[scale=0.9]{"Immagini workbook/1"}
		\end{figure}
				
		
		\begin{center}
			
			\vspace{1cm}
			
			\LARGE{\textbf{UNIVERSITÀ DEGLI STUDI DI PADOVA}}
			
			\vspace{0.5cm}
			
			\Large{\textbf{DIPARTIMENTO DI INGEGNERIA INDUSTRIALE}}
			
			\vspace{0.3cm}
			
			\large{\textbf{CORSO DI LAUREA MAGISTRALE IN INGEGNERIA AEROSPAZIALE}}
			\large{\textbf{CURRICULUM SPAZIALE}}
		\end{center}
		
		\vspace{3.5cm}
		
		\begin{center}
			\huge{\textbf{EXECUTIVE SUMMARY}}
			
			\hrulefill \\
			
			\LARGE{\textbf{ESPERIENZE DI LABORATORIO DEL CORSO DI}}\\
			\LARGE{\textbf{SATELLITE NAVIGATION}}
		\end{center}
		
		\vfill
		
		\begin{center}
			\Large{Anno Accademico 2023/2024}
		\end{center}
		
	\end{titlepage}
	
	\thispagestyle{empty}
	
	\cleardoublepage
	
	\thispagestyle{empty}
	
	\pagenumbering{Roman}
	\pagestyle{fancy}
	\fancyhead{}
	\renewcommand{\headrulewidth}{0pt}
	\fancyfoot[C]{\thepage}

	\clearpage
	
	\thispagestyle{empty}
	
	\listoffigures
	
	\listoftables
	
	\tableofcontents
	
	\clearpage
	
	\thispagestyle{empty}
	
	\mainmatter
	
	\chapter{Prima Esperienza di Laboratorio}
	
	\pagestyle{fancy}
	\fancyhead[RO,LE]{\thepage}
	\fancyhead[RE]{\leftmark}
	\fancyhead[LO]{\rightmark}
	\renewcommand{\headrulewidth}{0.5pt}
	\fancyfoot{}
	
	\section{Abstract}
	
	Con questa esperienza si vuole valutare l’effetto che la disposizione di più \textit{ground stations} emittenti un segnale GPS hanno sulla determinazione della posizione finale del ricevitore, detta \textit{fix}. In particolare, si vuole mostrare, utilizzando il metodo dei minimi quadrati, come una configurazione allineata delle \textit{ground station} abbia un effetto positivo sulla nuvola d’errore associata alla determinazione della posizione finale del suddetto \textit{fix}, garantendo un miglior risultato finale.
	
	
	\section{Introduzione}
	
	I GNSSs (\textit{Global Navigation Satellite Systems}) utilizzano delle tecniche di trilaterazione per determinare le posizioni, in una configurazione bidimensionale. In particolare, un ricevitore necessita di conoscere le coordinate del satellite (che in questo esempio sono paragonati a delle semplici \textit{ground stations}), gli errori di sincronizzazione degli orologi di bordo del satellite (noti come \textit{clock}) e di misurare la distanza tra il satellite e il ricevitore stesso, per poter trovare la sua posizione di quest'ultimo, nota come \textit{fix}.
		
	L'analisi è stata svolta considerando un problema bidimensionale diviso in 4 diversi scenari [tab. \ref{tab0}]:

	\begin{table}[H]
		\begin{center}
			\begin{tabular}{|c|c|c|}
				\hline
				& Errore di sincronizzazione dei \textit{clock} & Allineamento delle \textit{ground stations} \\
				\hline
				\textsl{Scenario 1} & No & No \\
				\hline
				\textsl{Scenario 2} & Si & No \\
				\hline
				\textsl{Scenario 3} & No & Si \\
				\hline
				\textsl{Scenario 4} & Si & Si \\
				\hline
			\end{tabular}
		    \caption{\textit{Proprietà degli scenari in analisi.}}
			\label{tab0}
		\end{center}
	\end{table}

	
	Per ciascuno di questi scenari è stato applicato il metodo dei minimi quadrati, che consente, partendo da dei dati di posizione iniziali, di ottenere una migliore soluzione andando a minimizzare una opportuna funzione interpolatrice.
	Per tenere conto del non allineamento delle \textit{ground stations} è stata inoltre implementata una metodologia che ricorre a un’analisi di covarianza, quindi a una funzione di costo.
	Così facendo si riesce a pesare il contributo che ciascuna \textit{ground station} porta all’errore complessivo.
	
	Verrà quindi valutato se l’introduzione di queste funzioni di costo comporti una diminuzione dell’errore finale.
	
	
	\section{Metodo di Analisi}
	
	In questa sezione sono presentati i codici MATLAB utilizzati per il calcolo del \textit{fix} di posizione e l'analisi di covarianza. In particolare l'esperienza prevede l'implementazione del metodo dei minimi quadrati sia con una opportuna matrice di peso che senza. Ciò è implementato nel codice principale \texttt{Fix$\_$GPS$\_$2D$\_$weighted.m}, che implementa l'algoritmo per la stima di posizione utilizzando il metodo dei minimi quadrati ricorsivo, con l'ausilio di codici secondari, nel modo seguente:
	
	\begin{figure}[H]
		\centering
		\subfloat[\textit{Metodo dei minimi quadrati non pesato.}]{{			\includegraphics[scale=0.85]{"Immagini workbook/Immagini esp1/algo"} }}
		\qquad
		\subfloat[\textit{Metodo dei minimi quadrati pesato.}]{{\includegraphics[scale=0.87]{"Immagini workbook/Immagini esp1/algo2"} }}
		\vspace{-0.3cm}
		\caption{\textit{Schematizzazione del metodo dei minimi quadrati implementato.}}
		\label{schemaalgo}
	\end{figure}
	
	\begin{enumerate}
		\item  Viene selezionato uno dei quattro scenari richiamando il codice secondario \texttt{scenario$\_$data.m}.
		
		Con questo codice si sono organizzati i dati sperimentali: questi consistono in misure di \textit{pseudorange} ottenute da due configurazioni diverse per quanto riguarda il numero e la posizione delle \textit{ground stations}. Per ognuna delle configurazioni è stato inizializzato uno scenario in cui è stato trascurato il termine $c\Delta \tau$ legato alla non sincronizzazione degli orologi e uno in cui invece è presente come variabile aggiuntiva. Si differenziano così le misure di \textit{pseudorange} dei quattro scenari in analisi [cfr. 1.2]. In questa istanza, viene anche definita la posizione "reale" o di "prima stima" del ricevitore $\underline{x}_0$.
		
		
		\item Segue il ciclo principale: dopo aver inizializzato le variabili, viene avviato un ciclo \textcolor{blue}{\texttt{while}} che itera l'algoritmo dei mninimi quadrati fintantoché non viene soddisfatta la condizione di arresto sulla norma del vettore degli incrementi $\Delta \underline{x}^{(k)}$ o viene raggiunto il numero massimo di iterazioni $k_{max}$.
		
		All'interno del ciclo vengono calcolate le misure di \textit{pseudorange} stimando la distanza tra il ricevitore e le \textit{ground stations}: da queste stime di distanza, si determina il vettore dei residui $\underline{y}^{(k)}$ tra le misure di \textit{pseudorange} effettive $\underline{\tilde{p}}$ e quelle calcolate $\underline{p}(\underline{x}^{(k)})$:
		
		\begin{equation}
			\underline{y}^{(k)} = \underline{\tilde{p}} - \underline{p}(\underline{x}^{(k)})
		\end{equation}
	
		Un'altra alternativa come condizione di arresto, complementare al raggiungimento massimo del numero di iterazioni, è di considerare un valore minimo di tolleranza $\chi^2 (\underline{x})$, definito come: $\chi^2 (\underline{x}) = \underline{y}^T\underline{y}$. Man mano che la norma del vettore degli incrementi $\Delta \underline{x}^{(k)}$ si riduce ad ogni passo di iterazione $k$, data la convergenza del metodo dei minimi quadrati, anche la norma del vettore degli scarti $\underline{y}^{(k)}$ si riduce e dunque anche il valore di $\chi^2(\underline{x}^{(k)})$. Infine, porre come condizione di arresto del ciclo \textcolor{blue}{\texttt{while}} o una condizione sulla norma minima di tolleranza del vettore degli incrementi o sul valore minimo di tolleranza di $\chi^2(\underline{x})$ è ininfluente: entrambi i metodi di arresto sono testimoni della convergenza dell'algoritmo del metodo dei minimi quadrati implementato.
		
		Si costruisce poi della matrice dei coefficienti $[H]$, così definita:
		
		\begin{equation}
			[H]^k = \left[ \frac{\partial p_i}{\partial x_j}(\underline{x}^{(k)}) \right]
		\end{equation}
	
		dove il pedice $i$ indica l'$i$-esima misura di \textit{psuedorange} rispetto alla $i$-esima \textit{ground station}, mentre il pedice $j$ indica il numero di variabili per il computo della stima della posizione finale del ricevitore.
		
		Si risolve quindi il sistema di equazioni lineari che come incognite ha gli incrementi $\Delta (\underline{x}^{(k)})$ da applicare alla posizione del ricevitore [fig. \ref{schemaalgo}], tenendo in considerazione la matrice di peso $[W]$, nel caso in cui si implementi la versione pesata dell'algoritmo dei minimi quadrati.
		
		Nella prima parte del codice secondario \texttt{covariance$\_$analysis$\_$weighted.m} è computata la matrice di peso $[W]$: con questo codice si esegue prima l'analisi di covarianza sull'incertezza della misura di posizione e in seguito sull’errore commesso durante la stima della posizione aggiornata. Ovvero, in primis, si analizza l'analisi di covarianza sugli scarti $\underline{y}^{(k)}$. Si differiscono i casi in cui si implementi o meno una matrice di peso $[W]$, che tenga conto di eventuali correlazioni nell'incertezza dei parametri previamente citati. La matrice di peso è così definita:
		
		\begin{equation}
			\begin{cases}
				[W] = [C_{pp}]^{-1} \\
				
				[C_{pp}] = \underline{y} \  \underline{y}^T
			\end{cases}
		\end{equation}
		
		Nel caso in cui si trascurino eventuali correlazioni tra le misure di \textit{pseudorange}, è sufficiente porre la matrice di peso $[W]$ pari a quella di identità.
		
		Infine, è verificata la condizione di arresto in base alla norma degli incrementi calcolati. La posizione del ricevitore viene aggiornata e vengono memorizzate le sue coordinate, se è rispettata la condizione di arresto.
	
	\item Il codice poi richiama il codice ausiliario \texttt{quadratic$\_$function.m}. Questo codice implementa le funzioni quadratiche $\chi^2(\underline{x}^{(k)})$ [cfr. pp.2] e $\Sigma^2(\Delta \underline{x}^{(k)})$ che rappresentano rispettivamente le funzioni di costo sugli scarti di \textit{pseudorange} e sugli incrementi di posizione. La funzione di costo sul valore dell'incremento di posizione è definita come: 
	
	$$\Sigma^2(\Delta \underline{x}^{(k)}) = (\underline{y}^{(k)}-[H]\Delta \underline{x})^T (\underline{y}^{(k)}-[H]\Delta \underline{x})$$
	
	L'obiettivo è che vengano minimizzate, in modo che il vettore degli incrementi $\Delta x$ sia minimo e che quindi la soluzione stimata finale di posizione del ricevitore sia il più fedele alla soluzione vera. Il codice genererà quindi un grafico di una funzione quadratica in modo da avere un riscontro visivo per quanto riguarda la minimizzazione dei del vettore degli incrementi [fig. \ref{chiquad}].
	
	\begin{figure}[H]
		\centering
		\subfloat[\textit{Migrazione del valore di $\chi^2 (\underline{x}^{(k)})$ verso il punto di minimo.}]{{			\includegraphics[scale=0.35]{"Immagini workbook/Immagini esp1/weight/scen4/ris415"} }}
		\qquad
		\subfloat[\textit{Posizione del valore di $\Sigma^2(\Delta \underline{x})$ all'ultima iterazione dell'algoritmo nel punto di minimo.}]{{\includegraphics[scale=0.35]{"Immagini workbook/Immagini esp1/weight/scen4/ris414"} }}
		\vspace{-0.3cm}
		\caption{\textit{Rappresentazioni grafiche della minimizzazione delle funzioni di costo $\chi^2 (\underline{x}^{(k)})$ e $\Sigma^2(\Delta \underline{x})$.}}
		\label{chiquad}
	\end{figure}

	\item Infine, vengono visualizzati graficamente i risultati del processo di localizzazione su un piano cartesiano [cfr. 1.4.1, fig. \ref{avvio}]. Con la seconda parte del codice ausiliario \texttt{covariance$\_$analysis$\_$weighted.m}, si computano i parametri dell'ellisse di incertezza (con un livello di confidenza del $99.7\%$) sull'ultima stima di posizione a partire dai valori della matrice di covarianza $[C_{xy}]$ sugli incrementi di posizione stimati dall'algoritmo definita come:
	
	\begin{equation}
		[C_{xy}] = \Delta \underline{x} \Delta \underline{x}^T = ([H][W][H])^{-1}
	\end{equation}

	Altrimenti, nel caso in cui si implementa il metodo dei minimi quadrati non pesati, i parametri dell'ellisse di incertezza si computano a partire sempre dai valori della matrice di covarianza $[C_{xy}]$, ma definita come:
	
	\begin{equation}
		[C_{xy}] = \sigma \cdot ([H]^T[H])^{-1}
	\end{equation}
	
	dove $\sigma$ è lo scarto quadratico medio sugli scarti $\underline{y}^{(k)}$.
				
	\end{enumerate}
	

	\section{Risultati ottenuti}
	
	Si decide di analizzare separatamente i risultati ottenuti implementando il metodo dei minimi quadrati. Tuttavia, è necessario anche differenziare i risultati ottenuti in funzione dell'allineamento o meno delle stazioni di terra emittenti.
	
	\subsection{Risultati ottenuti senza implementare una matrice di peso nel metodo dei minimi quadrati}
	
	Si confrontano i risultati ottenuti senza implementare una matrice di peso nel metodo dei minimi quadrati nel caso in cui si abbiano o meno le stazioni  di terra disposte lungo una direzione principale. Dunque, si confrontano, a titolo di esempio, lo \textsl{scenario 2} e lo \textsl{scenario 4}: quest'ultimo ha le \textit{ground stations} disposte lungo una direzione preferenziale [fig. \ref{avvio}, cfr. Tab. \ref{tab0}].
	
	\begin{figure}[H]
		\centering
		\subfloat[\textit{Disposizione delle ground stations non allineate dello \textsl{scenario 2}}]{{\includegraphics[scale=0.37]{"Immagini workbook/Immagini esp1/No weight/scen2/ris201"} }}
		\qquad
		\subfloat[\textit{Disposizione delle ground stations non allineate dello \textsl{scenario 4}}]{{\includegraphics[scale=0.37]{"Immagini workbook/Immagini esp1/No weight/scen4/ris401"} }}
		\vspace{-0.3cm}
		\caption{\textit{Confronto tra l'allineamento o meno delle ground stations.}}
		\label{avvio}
	\end{figure}
	
	Entrambi gli scenari tengono in considerazione la variabile di \textit{time delay} $c \Delta \tau$. Rispettivamente, \textsl{scenario 1} e \textsl{scenario 3} sono i corrispettivi scenari dove non è computato il valore della variabile $c \Delta \tau$ [tab. \ref{tab0}]: ciò presuppone un alleggerimento del costo computazione nell'implementazione del metodo dei minimi quadrati, ma implica una minore accuratezza nella stima del valore dello \textit{pseudorange}. Inoltre, le osservazioni che seguono per \textsl{scenario 2} e \textsl{scenario 4} sono analoghe rispettivamente per \textsl{scenario 1} e \textsl{scenario 3}. 
	Tuttavia, è molto più rilevante osservare le differenze dei risultati ottenuti se le \textit{ground stations} siano disposte lungo una direzione preferenziale o meno. Si confrontano quindi quanto la stima finale della posizione, computata con il metodo dei minimi quadrati, si scosti dalla posizione reale, ovvero la \textit{true position} [fig. \ref{avvio2}, ingrandimento fig. \ref{avvio}].
	
	\begin{figure}[H]
		\centering
		\subfloat[\textit{Ellissoide di incertezza e posizioni stimate dello \textsl{scenario 2} rispetto alla posizione reale}]{{\includegraphics[scale=0.3]{"Immagini workbook/Immagini esp1/No weight/scen2/ris203"} }}
		\qquad
		\subfloat[\textit{Ellissoide di incertezza e posizioni stimate dello \textsl{scenario 4} rispetto alla posizione reale}]{{\includegraphics[scale=0.3]{"Immagini workbook/Immagini esp1/No weight/scen4/ris403"} }}
		\vspace{-0.3cm}
		\caption{\textit{Confronto tra posizione finale stimata e posizione reale tra \textsl{scenario 2} e \textsl{scenario 4}.}}
		\label{avvio2}
	\end{figure}

	\vspace{-0.2cm}
	
	 Si nota subito una maggior precisione nella stima di posizione utilizzando il metodo dei minimi quadrati per il caso di \textsl{scenario 4} [fig. \ref{avvio2}.b], in cui le \textit{ground stations} siano allineate lungo una direzione preferenziale. Infatti, l'ultima stima di posizione è visivamente più vicina alla posizione reale e addirittura quest'ultima rientra nei confini dell'ellissoide di incertezza. Si ha anche un riscontro numerico [tab \ref{tab1}] e [tab. \ref{tab2}].
	 
	 \begin{table}[H]
	 	\begin{center}
	 		\begin{tabular}{|c|c|c|}
	 			\hline
	 			\textsl{Scenario 2} &  x [cm] & y [cm] \\
	 			\hline
	 			Posizione reale del ricevitore & 4.400 & 3.000 \\
	 			\hline
	 			Posizione stimata del ricevitore & 4.390 & 3.030 \\
	 			\hline
	 			Errore stimato & -0.010 & 0.030 \\
	 			\hline
	 		\end{tabular}
	 		\caption{\textit{Confronto tra posizione stimata e posizione reale de ricevitore in \textsl{scenario 2.}}}
	 		\label{tab1}
	 	\end{center}	
	 \end{table}
 
 	 \begin{table}[H]
\begin{center}
	\begin{tabular}{|c|c|c|}
		\hline
		\textsl{Scenario 4} &  x [cm] & y [cm] \\
		\hline
		Posizione reale del ricevitore & 4.400 & 3.000 \\
		\hline
		Posizione stimata del ricevitore & 4.403 & 3.009 \\
		\hline
		Errore stimato & 0.0014 & 0.0019 \\
		\hline
	\end{tabular}
\end{center}
 			\caption{\textit{Confronto tra posizione stimata e posizione reale de ricevitore in \textsl{scenario 4.}}}
 			\label{tab2}

 	\end{table}
	 
	In ultima battuta, si nota lievemente come l'ellisse di incertezza per \textsl{scenario 2} abbia più eccentricità rispetto a quello di \textsl{scenario 4}.
	 
	 \subsection{Risultati ottenuti implementando una matrice di peso nel metodo dei minimi quadrati}
	 
	 In modo analogo a quanto esposto precedentemente [cfr. 1.4.1], si analizzano solamente gli \textsl{scenario 2} e \textsl{scenario 4}. La disposizione delle \textit{ground stations} è medesima a [fig. \ref{avvio}].
	 
	 Nel computo della stima della posizione del ricevitore, viene implementata una matrice di peso $[W]$, affinchè si tengano conto delle incertezze nelle osservazioni compiute e di qualsiasi correlazione che possa sussistere tra queste. Dunque, la matrice di peso $[W]$ è definita come l'inversa della matrice di covarianza dei residui $[C_{pp}]$: $[W] = [C_{pp}]^{-1}$. Dunque, si confronta quanto la stima finale della posizione, computata con il metodo dei minimi quadrati pesato, si scosti dalla posizione reale, ovvero la \textit{true position} [fig. \ref{avvio3}].
	 
	 Anzitutto, si nota come il numero di iterazioni compiute dal metodo dei minimi quadrati per il computo della posizione finale stimata sia molto maggiore del caso in cui si implementi una matrice di peso [tab. \ref{tab3}].
	 
	 \begin{table}[H]
	 	\begin{center}
	 		\begin{tabular}{|c|c|c|}
	 			\hline
	 			& Iterazioni \textsl{Scenario 2} & Iterazioni \textsl{Scenario 4} \\
	 			\hline
	 			Algoritmo non pesato & 3 & 4 \\
	 			\hline
	 			Algoritmo pesato & 12 & 24 \\
	 			\hline
	 		\end{tabular}
 		 	\caption{\textit{Confronto numero di iterazioni compiute dal metodo dei minimi quadrati se pesato o meno.}}
 		    \label{tab3}
	 	\end{center}
	 \end{table}
	 
	 Ciò si spiega perchè l'incremento di posizione stimata rispetto al passo precedente è molto minore nel caso in cui si implementi una matrice di peso nell'algoritmo dei minimi quadrati: tanto che tale incremento risulti inferiore alla tolleranza di arresto imposta nell'algoritmo stesso. Significa che l'algoritmo computa posizioni stimate passo dopo passo lavorando sempre più di fino. Dunque, non sorprende se si nota una maggiore accuratezza nella stima della posizione finale [tab. \ref{tab5},\ref{tab4}].
	 
	 \begin{table}[H]
	 	\begin{center}
	 		\begin{tabular}{|c|c|c|}
	 			\hline
	 			\textsl{Scenario 2} &  x [cm] & y [cm] \\
	 			\hline
	 			Posizione reale del ricevitore & 4.400 & 3.000 \\
	 			\hline
	 			Posizione stimata del ricevitore & 4.391 & 3.025 \\
	 			\hline
	 			Errore stimato & -0.009 & 0.025 \\
	 			\hline
	 		\end{tabular}
	 		\caption{\textit{Confronto tra posizione stimata e posizione reale de ricevitore in \textsl{scenario 2} pesato.}}
	 		\label{tab5}
	 	\end{center}	
	 \end{table}

\begin{table}[H]
	\begin{center}
		\begin{tabular}{|c|c|c|}
			\hline
			\textsl{Scenario 4} &  x [cm] & y [cm] \\
			\hline
			Posizione reale del ricevitore & 4.400 & 3.000 \\
			\hline
			Posizione stimata del ricevitore & 4.386 & 3.019 \\
			\hline
			Errore stimato & -0.003 & 0.009 \\
			\hline
		\end{tabular}
		\caption{\textit{Confronto tra posizione stimata e posizione reale de ricevitore in \textsl{scenario 4} pesato.}}
		\label{tab4}
	\end{center}	
\end{table}

	Si nota come la maggiore eccentricità dell'ellisse di incertezza del caso di \textsl{scenario 2} venga maggiormente intensificata rispetto al caso non pesato [fig. \ref{avvio2}]. L'eccentricità dell'ellisse di incertezza per il caso \textsl{scenario 4} rimane pressochè invariata in confronto al corrispettivo non pesato [fig. \ref{avvio2}].
	
	\begin{figure}[H]
		\centering
		\subfloat[\textit{Ellissoide di incertezza e posizioni stimate dello \textsl{scenario 2} rispetto alla posizione reale}]{{\includegraphics[scale=0.3]{"Immagini workbook/Immagini esp1/weight/scen2/ris213"} }}
		\qquad
		\subfloat[\textit{Ellissoide di incertezza e posizioni stimate dello \textsl{scenario 4} rispetto alla posizione reale}]{{\includegraphics[scale=0.3]{"Immagini workbook/Immagini esp1/weight/scen4/ris413"} }}
		\vspace{-0.3cm}
		\caption{\textit{Confronto tra posizione finale stimata e posizione reale tra \textsl{scenario 2} e \textsl{scenario 4}.}}
		\label{avvio3}
	\end{figure}

	\vspace{-0.7cm}
	
	\section{Conclusioni}
	
	Anzitutto, si conferma quanto fosse previsto: il metodo dei minimi quadrati pesato offre un computo iterativo della posizione stimata del ricevitore molto più accurata se non fosse pesato. La conferma è nello scostamento del \textit{fix} finale dalla \textit{true position} del ricevitore [Tab. \ref{tab5}, \ref{tab4}].
	
	Si riscontra un incremento nell'eccentricità dell'ellisse di incertezza nel caso si passi dalla versione non pesata a quella pesata, sopratto nel caso di \textsl{scenario 2} [fig. \ref{avvio3}]: questo riconferma che l'accuratezza della posizione finale del \textit{fix} è migliore nel caso in cui si implementi un metodo dei minimi quadrati pesato. Tuttavia, considerando le implementazioni del metodo dei minimi quadrati pesati, l'eccentricità dell'ellisse di incertezza in \textsl{scenario 2} è maggiore del caso di \textsl{scenario 4}: questo suggerisce che una disposizione casuale delle \textit{ground stations} implichi una miglior accuratezza nella stima della posizione finale del \textit{fix}. Ciò è dovuto al fatto che gli errori commessi nella misura dei \textit{pseudorange} hanno una scala perfettamente commensurabile e confrontabile con la sensibilità dello strumento con cui sono state stimate le misure citate. Dunque, nel computo della matrice di peso, possono crearsi delle situazioni particolari di singolarità, in cui l'eccentricità dell'ellisse di incertezza incrementi, alludendo ad una minor varianza lungo una direzione preferenziale, nel computo della posizione finale del \textit{fix}.
	
	\section{Bibliografia}
	
	[1] - Andrea Valmorbida, PhD, \textit{Appunti del Corso di "Satellite Navigation"}, Università degli Studi di Padova, Dipartimento di Ingegneria Industriale (dii), 2024.
	
	\chapter{Seconda Esperienza di Laboratorio}
	
	\section{Abstract}
	
	Con questa esperienza si vuole valutare la precisione e l'affidabilità nel tempo della stima delle posizioni dei satelliti in orbita, appartenenti a determinate costellazioni GNSS, in particolare GPS e GLONASS. Le posizioni di ricavano da due tipologie diverse di dati:
	
	\begin{enumerate}
		\item \textit{Navigation Broadcast Message};
		
		\item \textit{Precise Products}.
	\end{enumerate}
	
	Una volta ricavate le informazioni necessarie, le orbite dei satelliti GNSS sono poi visualizzate
	con una rappresentazione tridimensionale. Successivamente si esegue un confronto tra le orbite ottenute con i due tipi di dati per andare a valutarne l’accuratezza.
	Infine è generato lo \textit{skyplot} relativo alla posizione di un dato utente.
	
	\section{Introduzione}
	
	Manca questa parte
	
	\section{Metodo di Analisi}
	
	L'esercitazione ha l'obiettivo di computare le posizioni dei satelliti e gli errori sui loro clock usando due diverse tipologie di dati:
	
	\begin{enumerate}
		\item Dati trasmessi dal \textit{Broadcast Navigation Message};
		
		\item \textit{Precise Product} elaborati e scaricati dal \textit{International GNSS Service} (IGS);
	\end{enumerate}

	Le posizioni ottenute sono quindi usate per visualizzare gli archi di orbita percorsi durante il tempo di simulazione in uno scenario 3D.
	Il focus dell'esercitazione si sposta poi sulla comparazione dei risultati delle due diverse tipologie di dati elaborati, in modo da riscontrare una divergenza dei dati dopo un tempo di interpolazione dell'ordine dell'ora.
	In conclusione si evidenziano gli \textit{skyplot} dei satelliti e relativa visualizzazione tridimensionale delle orbite osculatrici.
	
	I tre script principali al loro interno richiamano alcune funzioni già presenti in MATLAB per operazioni di manipolazione dei dati che sono di scarso interesse al fine di questa esercitazione e pertanto la trattazione della metodologia di analisi si concentra principalmente sui codici principali. Lo svolgimento di ogni punto prevede l'uso esclusivo di MATLAB per la generazione di script e relative \texttt{function} che leggano i dati dai file RINEX e SP3.

	
	\subsection{Broadcast Orbits}
	
	Per l'analisi delle \textit{Broadcast Orbits} si è stilato il codice \texttt{Broadcast$\_$orbits.m}. Le prime righe del codice hanno l'importante compito di configurare le impostazioni iniziali dello script, in particolare si specificano:
	
	\begin{itemize}[-]
		\item il file RINEX dal quale ottenere i dati;
		
		\item la costellazione di satelliti da analizzare, che in questo studio è o quella GPS o quella GLONASS;
		
		\item gli identificativi, in codice "ID", dei satelliti che si vogliono studiare;
		
		\item il tempo di interpolazione.
	\end{itemize}

	Si estraggono i dati delle effemeridi per la costellazione selezionata e, più nel dettaglio, si estraggono anche le effemeridi per un numero selezionato di satelliti specifici alla costellazione precedentemente scelta. Questo si esegue estraendo le informazioni necessarie dal file RINEX dato a disposizione (BRDC00IGS$\_$R$\_$2024031000 0$\_$01D$\_$MN.rnx), tramite l'implentazione di MATLAB \texttt{function} caratteristiche del \textit{Navigation Toolbox}: \texttt{rinexread} e \texttt{rinexinfo}.
	Successivamente si inizializza un doppio ciclo \texttt{\textcolor{blue}{for}} per il calcolo delle coordinate $[X,Y,Z]$ delle posizioni orbitali in un sistema di riferimento CTS (anche noto come ECEF, ovvero \textit{Earth Centered, Earth Fixed}), per ciascuno dei satelliti in interesse e per ogni intervallo in cui è stato discretizzato il tempo di interpolazione. Tutto ciò si esegue a partire dai dati delle effemeridi precedentemente ottenute.
	
	Questa operazione viene fatta dopo aver discriminato la costellazione in esame, se GPS o GLONASS, in modo che il computo delle coordinate avvenga tramite il richiamo della rispettiva MATLAB \texttt{function} \texttt{GPS$\_$coordinates.m} o \texttt{GLONASS$\_$coordinates}. Nel dettaglio:
	
	\begin{itemize}[-]
		\item La MATLAB \texttt{function} \texttt{GPS$\_$coordinates.m} implementa il seguente algoritmo per ottenere le coordinate cartesiane in un sistema di riferimento CTS, che ha come base il sistema geodetico di riferimento WGS-84, di un satellite al tempo voluto $t_k$:
		
		\begin{enumerate}
			\item Dal file RINEX si estraggono i seguenti dati di effemeridi [fig. \ref{fig:efembrd}]:
			
			\begin{figure}[H]
				\centering
				\includegraphics[scale=0.47]{"Immagini workbook/Immagini esp2/efembrd"}
				\caption{\textit{Lista di valori che caratterizzano le effemeridi di un satellite di una costellazione GPS (ma anche Galileo e BeiDou) trasmessi dal Broadcast Navigation Message}}
				\label{fig:efembrd}
			\end{figure}
		
			\item Calcolo del tempo $t_k$ passato dall'epoca di riferimento delle effemeridi $t_{oe}$:
			
			\begin{equation}
				t_k = t - t_{oe}
			\end{equation}
		
			\item Calcolo dell'anomalia media al tempo $t_k$:
			
			\begin{equation}
				M_k = m_0 + \left( \sqrt{\frac{\mu}{a^3}} + \Delta n \right) t_k 
			\end{equation}
		
			\item Calcolo dell'anomalia eccentrica al tempo $t_k$ risolvendo iterativamente l'equazione di Keplero:
			
			\begin{equation}
				M_k = E_k - e \sin E_k
			\end{equation}
		
			\item Calcolo dell'anomalia vera al tempo $t_k$:
			
			\begin{equation}
				\nu_k = \arctan \left( \frac{\sqrt{1-e^2}\sin E_k}{\cos E_k - e} \right) 
			\end{equation}
			
			\item Calcolo dell'argomento di latitudine $u_k$ al tempo $t_k$ a partire dalla conoscenza dell'argomento di perigeo $\omega$, dell'anomalia vera $\nu_k$. Tuttavia, a causa della non perfetta sfericità del campo gravitazionale terrestre, delle perturbazioni Luni-Solari e delle perturbazioni dovute all'effetto della pressione di radiazione solare, i valori di argomento di perigeo $\omega$ e di anomalia vera $\nu$ variano nel tempo e queste variazioni sono incorporate nei fattori correttivi $c_{uc}$ e $c_{us}$. Segue quindi:
			
			\begin{equation}
				u_k = \omega + \nu_k + c_{uc} \cos 2(\omega + \nu_k) + c_{us} \sin 2(\omega + \nu_k)
			\end{equation}
		
		\item Calcolo della distanza radiale $r_k$ al tempo $t_k$, tenendo conto delle rispettive correzioni dovute alle perturbazioni orbitali, ovvero $c_{rc}$ e $c_{rs}$:
		
		\begin{equation}
			r_k = a(1-e \cos E_k ) + c_{rc} \cos 2(\omega + \nu_k) + c_{rs} \sin 2(\omega + \nu_k)
		\end{equation}
		
		\item Calcolo dell'inclinazione del piano orbitale $i_k$ al tempo $t_k$, a partire dall'inclinazione $i_0$ di riferimento all'inizio dell'epoca di riferimento $t_{oe}$ e successiva variazione nel tempo $\dot{i}$, corretta dalle perturbazioni orbitali con i rispettivi parametri $c_{ic}$ e $c_{is}$:
		
		\begin{equation}
			i_k = i_0 + \dot{i}t_k + + c_{ic} \cos 2(\omega + \nu_k) + c_{is} \sin 2(\omega + \nu_k)
		\end{equation}
		
		\item Calcolo della longitudine del nodo ascendente $\lambda_k$ al tempo $t_k$, facendo riferimento al meridiano di Greenwich. Si calcola a partire dal valore di RAAN (\textit{Right Ascenscion of Ascending Node}) $\Omega_0$ iniziale all'inizio della settimana di riferimento dell'epoca considerata $t_{oe}$, corretta dalla precessione del nodo ascendente $\dot{\Omega}$ decurtata della rotazione terrestre $\omega_E$ e dall'arco di longitudine percorso dalla Terra nel tempo $t_{oe}$:
		
		\begin{equation}
			\lambda_k = \Omega_0 + (\dot{\Omega} - \omega_E)t_k - \omega_E t_{oe}
		\end{equation}
	
		\item Infine, calcolo delle coordinate cartesiane nel sistema di riferimento CTS, applicando tre rotazioni successive al vettore di posizione radiale:
		
		\begin{equation}
			\begin{aligned}
				\begin{bmatrix}
					X \\
					Y \\
					Z
				\end{bmatrix}
			\end{aligned} = [R_3 (-\lambda_k)] [R_1 (-i_k)]  [R_3 (- u_k)] \begin{aligned}
			\begin{bmatrix}
				r_k \\
				0 \\
				0
			\end{bmatrix}
		\end{aligned} 
		\end{equation}
		\end{enumerate}
	
		\item La MATLAB \texttt{function} \texttt{GLONASS$\_$coordinates.m} serve a ricavare la posizione del satellite GLONASS scelto al tempo $t_k$ desiderato seguendo la seguente metodologia:
		
		\begin{enumerate}
			\item Dal file RINEX si estraggono i seguenti dati di effemeridi [fig. \ref{fig:efemglo}]
			
			\begin{figure}[H]
				\centering
				\includegraphics[scale=0.75]{"Immagini workbook/Immagini esp2/efemglo"}
				\caption{\textit{Lista di valori che caratterizzano le effemeridi di un satellite di una costellazione GLONASS (ma anche Galileo e BeiDou) trasmessi dal Broadcast Navigation Message}}
				\label{fig:efemglo}
			\end{figure}
		
			Si osserva come le effemeridi trasmesse dai satelliti GLONASS non contengono i parametri orbitali kepleriani e le loro relative correzioni, bensì contengono le posizioni e le velocità iniziali dei satelliti al tempo di riferimento di epoca $t_e$, caratterizzando così il vettore di stato iniziale del satellite stesso, espresso nel sistema geodetico di riferimento PZ-90. Inoltre, le perturbazioni orbitali dovute principalmente all'interazione Luni-Solare sono espresse in termini di accelerazioni.
			
			\item Si esegue l'integrazione numerica delle equazioni differenziali del moto dei satelliti direttamente in un sistema di riferimento ECEF. L'integrazione numerica delle equazioni del moto è compiuto tramite la MATLAB \texttt{function} \texttt{ode45} e va eseguita per intervalli discreti di tempo inferiori a 15 minuti. Siccome il sistema di riferimento ECEF non è inerziale, le accelerazioni dovute alla forza di Coriolis e centrifuga devono essere prese in considerazione.
			
			Si esplicitano le formulazioni del vettore di stato $\underline{y}(t)$ di un satellite GLONASS e della sua derivata prima $\underline{\dot{y}}(t)$, necessarie per l'integrazione numerica tramite la MATLAB \texttt{function} \texttt{ode45}:
			
			\begin{equation}
				\begin{aligned}
					\underline{y}(t) = \begin{bmatrix}
						x(t)\\
						\dot{x}(t)\\
						y(t)\\
						\dot{y}(t)\\
						z(t)\\
						\dot{z}(t)
					\end{bmatrix}
				\end{aligned} \ \ , \ \ \begin{aligned}
				\underline{\dot{y}}(t) = \begin{bmatrix}
				y_2\\
				-\frac{\mu y_1}{r^3} + \frac{3}{2} J_2 \frac{\mu r_E^2}{r^5} y_1 \left( 1 - \frac{5y_5^2}{r^2} \right) + \omega_E^2y_1 + 2 \omega_E y_4 + \ddot{y}_{1LS} \\
				y_4 \\
				- \frac{\mu y_3}{r^3} + \frac{3}{2} J_2 \frac{\mu r_E^2}{r^5}y_3 \left( 1 - \frac{5 y_5^2}{r^2} \right) + \omega_E^2 y_3 - 2 \omega_E y_2 + \ddot{y}_{3LS} \\
				y_6\\
				- \frac{\mu y_5}{r^3} + \frac{3}{2} J_2 \frac{\mu r_E^2}{r^5}y_5 \left( 3 - \frac{5 y_5^2}{r^2} \right)  + \ddot{y}_{5LS}
			\end{bmatrix}
			\end{aligned}
			\end{equation}
		
			In particolare, le formulazioni per il vettore $\underline{\dot{y}}(t)$ sono state esplicate in una MATLAB \texttt{function} apposita: \texttt{GLONASS$\_$eq.m}, che poi è richiamata negli input di \texttt{ode45} per procedere all'integrazione numerica.
		
		\item Infine, si esegue una trasformazione dal sistema geodetico di riferimento PZ-90 a quello WGS-84 e si estrapolano solo gli elementi non derivati nel tempo del vettore di stato al fine di ricavare le coordinate cartesiane del satellite GLONASS di interesse nel sistema di riferimento CTS.
		\end{enumerate}
	\end{itemize}
	
	Successivamente, le posizioni ottenute sono memorizzate e immesse nel \textit{viewer} di \texttt{satelliteScenario}, al fine di visualizzare l'evoluzione temporale tridimensionale delle orbite dei satelliti considerati rispettivamente delle costellazioni GPS [fig. \ref{gpssc}] e GLONASS [fig. \ref{glosc}] nel tempo di interpolazione.
	
	\begin{figure}[H]
		\centering
	    {{\includegraphics[scale=0.27]{"Immagini workbook/Immagini esp2/gps3d1"}}}
		\qquad
		{{\includegraphics[scale=0.27]{"Immagini workbook/Immagini esp2/gps3d2"} }}
		\vspace{-0.3cm}
		\caption{\textit{Visualizzazione tridimensionali delle orbite dei satelliti SAT1, SAT15 e SAT30 della costellazione GPS.}}
		\label{gpssc}
	\end{figure}

	\begin{figure}[H]
		\centering
		{{\includegraphics[scale=0.27]{"Immagini workbook/Immagini esp2/glo3d1"}}}
		\qquad
		{{\includegraphics[scale=0.27]{"Immagini workbook/Immagini esp2/glo3d2"} }}
		\vspace{-0.3cm}
		\caption{\textit{Visualizzazione tridimensionali delle orbite dei satelliti SAT1, SAT12 e SAT24 della costellazione GLONASS.}}
		\label{glosc}
	\end{figure}


	\subsection{Precise Orbits}
	Per l'analisi delle \textit{Precise Orbits} si è stilato il codice \texttt{Precise$\_$orbits.m}. Le prime righe del codice hanno il compito di configurare le impostazioni iniziali dello script, in particolare si specificano:
	
	\begin{itemize}[-]
		\item il file SP3 dal quale ottenere i dati;
		
		\item la costellazione di satelliti da analizzare, che in questo studio o è quella GPS o è quella GLONASS;
		
		\item gli identificativi, in codice "ID", dei satelliti che si vogliono studiare;
		
		\item il tempo di interpolazione.
	\end{itemize}

	Si estraggono le informazioni delle coordinate cartesiane delle posizioni dei satelliti direttamente dal file SP3 dato (ESA0MGNFIN$\_$20240310000$\_$01D$\_$05M$\_$ORB.SP3), tramite la MATLAB \texttt{function} \texttt{read$\_$sp3$\_$mul- ticonstellation.m}. Le coordinate cartesiane sono raccolte ad intervalli di 15 minuti.
	
	Si possono quindi memorizzare e immettere le posizioni cartesiane dei satelliti di interesse nel \textit{viewer} di \texttt{satelliteScenario} [fig. \ref{sp3gps}, \ref{sp3glo}].
	
	\begin{figure}[H]
		\centering
		\includegraphics[scale=0.27]{"Immagini workbook/Immagini esp2/gpssp31"}
		\caption{\textit{Visualizzazione delle orbite dei SAT1, SAT15 e SAT30 della costellazione GPS.}}
		\label{sp3gps}
	\end{figure}

	\begin{figure}[H]
		\centering
		\includegraphics[scale=0.27]{"Immagini workbook/Immagini esp2/gpssp31"}
		\caption{\textit{Visualizzazione delle orbite dei SAT1, SAT12 e SAT24 della costellazione GLONASS.}}
		\label{sp3glo}
	\end{figure}
	
	
	%\begin{figure}[H]
	%	\centering
	%	\subfloat[\textit{Visualizzazione delle orbite dei SAT1, SAT15 e SAT30 della costellazione GPS.}]{{\includegraphics[scale=0.27]{"Immagini workbook/Immagini esp2/gpssp31"}}}
	%	\qquad
	%	\subfloat[\textit{Visualizzazione delle orbite dei SAT1, SAT12 e SAT24 della costellazione GLONASS.}]{{\includegraphics[scale=0.27]{"Immagini workbook/Immagini esp2/glosp31"} }}
	%	\vspace{-0.3cm}
	%	\caption{\textit{Visualizzazione delle orbite di tre satelliti delle costellazioni GPS e GLONASS a partire dai \textit{Precise Products} ricavati dal file SP3 dato.}}
	%	\label{satscsp3}
	%\end{figure}
	
	L'operazione interessante che si può svolgere con i dati di \textit{Precise Products} ricavati dal file SP3 di interesse è la seguente: è possibile ricavare le posizioni dei satelliti per epoche temporali successive, anche a partire con solo i dati di posizioni ricavati all'inizio e alla fine dell'intervallo di acquisizione di 15 minuti. Queste posizioni di epoche temporali successive si ricavano implementando un'interpolazione lagrangiana polinomiale di ottavo o di decimo grado, affinché si abbiano precisioni sulla posizione dell'ordine del centimetro. Tale operazione è svolta dalla specifica MATLAB \texttt{function} \texttt{interp$\_$precise$\_$orbits.m}.
	
	Si confrontano quindi le posizioni previste nel futuro del polinomio interpolante con le posizioni effettivamente ricavate dall'analisi dei \textit{Precise Products}, precedentemente ottenute dalla lettura del file SP3 dato.\\
	
		\begin{figure}[H]
		\centering
		\includegraphics[scale=0.35]{"Immagini workbook/Immagini esp2/interp111"}
		\caption{\textit{Visualizzazione delle orbite dei SAT1, SAT12 e SAT24 della costellazione GLONASS.}}
		\label{interp111}
	\end{figure}
	
	Si nota come la bontà della predizione delle posizioni date dal polinomio interpolante è valido fino a qualche ora, e dopo quel margine di tempo, si osserva una divergenza sostanziale dalle posizioni effettive ottenute da nuovi campioni di dati di \textit{Precise Orbits}.
	
	Inotre, si osserva come la coordinata cartesiana computata è la prima si osserva come abbia un valore pressoché nullo: questo indica che il satellite considerato per cui si è svolta l'operazione di interpolazione per stimare le sue posizioni orbitali successive si muove lungo un piano orbitale su cui giace l'asse $x$ del sistema di riferimento considerato, ovvero quello ECEF.
	
	\clearpage
	
	\section{Risultati}
	
	Per analizzare i risultati ottenuti, è utile confrontare lo scostamento delle posizioni dei satelliti in orbita ottenute con il \textit{Broadcast Navigation Message} e con i \textit{Precise Products}, con il numero minimo di informazioni necessarie.
	
	Il \textit{Navigation Broadcasted Message} è trasmesso in forma continua e quindi si ha un rinnovo delle informazioni delle posizioni dei satelliti in modo continuo, mentre è possibile conoscere le posizioni dei satelliti anche fino a qualche ora nel futuro con solo le informazioni presenti in un intervallo di campionamento di 15 minuti dei \textit{Precise Products}.
	
	Dunque, si confrontano le posizioni orbitali dei satelliti stimati con le \textit{broadcasted ephemerisis} e il polinomio interpolatore ricavato da una sola acquisizione di \textit{Precise Prodcuts}.
	
	Nello specifico, si è stilato il codice \texttt{Compare$\_$Broadcast$\_$Precise$\_$orbits.m}, in cui si esegue il confronto sopracitato in maniera specifica per il SAT4 della costellazione GPS [fig. \ref{diff}].
	
	
	\begin{figure}[H]
		\centering
		\subfloat[\textit{Differenza nel computo delle coordinate cartesiane della posizione orbitale del SAT4 della costellazione GPS nel tempo di interpolazione scelto.}]{{\includegraphics[scale=0.32]{"Immagini workbook/Immagini esp2/diff1"}}}
		\qquad
		\subfloat[\textit{Particolare della differenza nel computo delle coordinate cartesiane della posizione orbitale del SAT4 della costellazione GPS}]{{\includegraphics[scale=0.32]{"Immagini workbook/Immagini esp2/diff2"} }}
		\vspace{-0.3cm}
		\caption{\textit{Differenza nel computo delle coordinate cartesiane della posizione orbitale del SAT4 della costellazione GPS nel tempo di interpolazione scelto e relativo particolare.}}
		\label{diff}
	\end{figure}

	Si osserva come la differenza è dell'ordine dell'unità del metro e di suoi sottomultipli fintanto quando ha validità l'interpolazione delle posizioni orbitali dei satelliti con in \textit{Precise Products}: infatti, la differenza tra i due computi delle posizioni diverge in corrispondenza della divergenza del polinomio interpolatore stesso.
	
	\section{Conclusioni}
	
	Si nota come entrambe le metodologie per stimare le posizioni orbitali dei satelliti di una costellazione GNSS abbiano dei vantaggi e dei svantaggi:
	
	\begin{enumerate}
		\item Effemeridi trasmesse dal \textit{Broadcast Navigation Message}
		
		\begin{itemize}[-]
			\item Vantaggi:
			
			\begin{enumerate}
				\item Il \textit{Broadcast Navigation Message} trasmesso da un satellite veicola informazioni solo sulle proprie effemeridi e non su quelle di tutti gli altri satelliti della costellazione: questo permette un messaggio di navigazione più corto, che garantisce una ripetibilità di trasmissione altrettanto alta.
				
				\item Il \textit{Broadcast Navigation Message} è trasmesso in forma continua nel segnale trasmesso dal satellite di interesse, con i parametri aggiornati continuamente.
			\end{enumerate}
		
			\item Svantaggi:
			
			\begin{enumerate}
				\item Le effemeridi trasmesse dal \textit{Broadcast Navigation Message} hanno un'accuratezza nella stima della posizione orbitale del satellite dell'unità del metro.
				
				\item Si ha una certa complessità nel computo delle coordinate cartesiane delle posizioni orbitali dei satelliti in orbita, in quanto si devono eseguire algoritmi specifici (come per le costellazioni quali GPS, Galileo e BeiDou) per passare da valori di parametri orbitali alle coordinate cartesiane sopracitate in un sistema di riferimento CTS (anche noto come ECEF); oppure si devono eseguire delle integrazioni numeriche per ottenere i valori finali delle posizioni (come avviene nell'analisi dellla costellazione GLONASS).
				
				\item Si possono computare le posizioni solo dei satelliti in vista. Il numero di satelliti in vista è messo in evidenza in un grafico di tipo \textit{skyplot}: tale grafico è la proiezione bidimensionale di un sistema di riferimento ENU, centrato su un punto di interesse sul geoide di riferimento. Questo grafico traccia la traiettoria del satellite passante nella volta celeste visibile all'osservatore, che determina il centro del grafico stesso. A titolo di esempio, si evidenzia lo \textit{skyplot} per i satelliti in vista della costellazione GPS [fig. \ref{fig:skyplotbrd}].
				
				\begin{figure}[H]
					\centering
					\includegraphics[scale=0.55]{"Immagini workbook/Immagini esp2/skyplotbrd"}
					\caption{\textit{Skyplot dei satelliti in vista della costellazione GPS}}
					\label{fig:skyplotbrd}
				\end{figure}
			
				Da notare come le elevazioni più basse sono oscurate da un opportuno \textit{mask angle}: ciò è dovuto al fatto che si ha una buona acquisizione del segnale a partire da un'elevazione minima di $15^o \div 18^o$.
				
				Se si vuole visualizzare solo il numero di satelliti in vista nel periodo temporale di analisi, si può esplicare il seguente grafico [fig. \ref{satvista}]:
				
				\begin{figure}[H]
					\centering
					\includegraphics[scale=0.55]{"Immagini workbook/Immagini esp2/satvista"}
					\caption{\textit{Numero dei satelliti in vista della costellazione GPS, con orbite computate con Navigation Broadcast Message.}}
					\label{satvista}
				\end{figure}
			
			Da notare come si hanno sempre almeno più di quattro satelliti in vista: questo garantisce il processo di quadrilaterazione affinchè si possa determinare la posizione dell'osservatore.
				
			\end{enumerate}
		\end{itemize}
	
		\item Posizioni ricavate dai \textit{Precise Products}
		
		\begin{itemize}[-]
			\item Vantaggi:
			
			\begin{enumerate}
				\item Le posizioni orbitali calcolate a partire dai \textit{Precise Products} hanno un'accuratezza dell'ordine del centimentro.
				
				\item Non sono necessari algoritmi o integrazioni numeriche per avere le coordinate cartesiane delle posizioni orbitali dei satelliti: i \textit{Precise Products} già esplicano tali valori.
				
				\item Si possono calcolare la posizioni di tutti i satelliti della costellazione di interesse e non solo quelli in vista, in quanto i \textit{Precise Products} sono dati pre-elaborati.
				
				Tuttavia, si presenta lo \textit{skyplot} anche per le traiettorie dei satelliti computati con i \textit{Precise Products} [fig. \ref{fig:skyplotsp3}]:
				
				\begin{figure}[H]
					\centering
					\includegraphics[scale=0.55]{"Immagini workbook/Immagini esp2/skyplotsp3"}
					\caption{\textit{Skyplot dei satelliti in vista della costellazione GPS, con orbite computate con Precise Products.}}
					\label{fig:skyplotsp3}
				\end{figure}
			
				In modo analogo, si hanno il numero di satelliti in vista nel tempo di validità del file SP3 considerato per la costellazione GPS
				
				\begin{figure}[H]
					\centering
					\includegraphics[scale=0.55]{"Immagini workbook/Immagini esp2/satvistasp3"}
					\caption{\textit{Numero dei satelliti in vista della costellazione GPS, con orbite computate con Precise Products.}}
					\label{satvistasp3}
				\end{figure}
				
			\end{enumerate}
		
			\item Svantaggi:
			
			\begin{enumerate}
				\item I \textit{Precise Products} non forniscono le coordinate cartesiane delle posizioni orbitali dei satelliti in maniera discreta, ad intervalli di 15 minuti.
				
				\item La trasmissione dei \textit{Precise Pridcuts} non è continua e in tempo reale, ma sono pubblicati con un ritardo rispetto all'osservazione fatta, con intervalli di aggiornamento che spaziano da qualche ora fino a cadenza settimanale, a seconda della precisione con cui si vogliono avere.
				
				\item È possibile stimare la posizione orbitale del satellite di interesse solo con i \textit{Precise Products} di una acquisizione nominale di 15 minuti, tramite un opportuno polinomio interpolatore. Tuttavia, la validità temporale nel futuro delle posizioni stimate dal polinomio interpolatore è dell'ordine dell'ora.
			\end{enumerate}
		\end{itemize}
	\end{enumerate}

	Inoltre, la differenza nel computo delle coordinate cartesiane delle posizioni orbitali con le due metodologie [cfr. 2.4, fig. \ref{diff}] è principalmente governata dall'accuratezza peggiore registrata da ogni singola metodologia: infatti, finché si è nell'intervallo temporale in cui il polinomio interpolatore ha validità, la differenza nel computo delle posizioni con i due metodi è dello stesso ordine di grandezza del metodo con peggior accuratezza tra i due, ovvero quello che fa uso del \textit{Broadcast Navigation Message}; successivamente, quando si esce dall'intervallo temporale di validità del polinomio interpolatore, si osserva come la differenza ha lo stesso ordine di grandezza della divergenza registrata nel computo delle posizioni orbitali del polinomio interpolatore stesso.
	
	\section{Bibliografia}
	
	[1] - Andrea Valmorbida, PhD, \textit{Appunti del Corso di "Satellite Navigation"}, Università degli Studi di Padova, Dipartimento di Ingegneria Industriale (dii), 2024.
	
	\chapter{Terza Esperienza di Laboratorio}
	
	\section{Abstract}
	
	L'obiettivo di questa esperienza è di ottenere la posizione del ricevitore usando misure di pseudorange, adoperando le informazioni ottenute sia dalle effemeridi trasmesse dai satelliti, sia da quelle ottenute dai prodotti precisi.
	
	\section{Introduzione}
	
	Manca
	
	\section{Metodo di Analisi}
	
	Il codice usato per ottenere la posizione finale del ricevitore è \texttt{GNSS$\_$SPP$\_$main.m}, nel quale è possibile analizzare solo una costellazione GNSS alla volta, specificando il suo nome all'inizio.
	
	Segue poi la definizioni delle costanti fisiche principali come:
	
	\begin{itemize}[-]
		\item la velocità della luce, pari a $c = 299792458 m/s$;
		
		\item la costante di gravitazione per la Terra, facente riferimento al modello geodetico di riferimento WGS-84, pari a $\mu = 3.986005 \cdot 10 ^{14}m^3/s^2$;
		
		\item il raggio terrestre medio, pari a $R_E = 6'378'137m$;
		
		\item la velocità angolare di rotazione terrestre media, pari a $\omega_E = 7.2921151467 \cdot 10^{-5} rad/s$, secondo i modelli usati dalle costellazioni GPS e Galileo;
		
		\item la costante di accelerazione gravitazionale terrestre, pari a $g = 9.80665m/s^2$;
		
		\item i \textit{leap seconds} pari a $18$ secondi. Prima di esplicare l'origine di questi, è bene  ricordare che esiste una discrepanza registrata tra il tempo scandito dal moto della Terra rispetto al Sole, noto come \textit{Universal Time} (UT), e quello scandito da un sistema di orologi atomici, gestiti da il \textit{Bureau International de l'Heure} (BIH) e noto come \textit{International Atomic Time} (TAI). Siccome la rotazione della Terra non è costante, il tempo scandito dal sistema UT non è uniforme e accumula un \textit{drift} rispetto al tempo registrato dal sistema TAI. Tuttavia, si definisce lo \textit{Corrdinated Universal Time} (UTC), che è un compromesso tra il tempo registrato dal sistema TAI e dal sistema UT. Lo UTC segue il tempo scandito dal sistema TAI, a cui viene "sottratto" periodicamente un certo numero di secondi, detti appunto \textit{leap seconds}, in modo che il tempo registrato abbia una discrepanza al massimo di 0,9 secondi rispetto a quello segnato dal sistema UT [fig. \ref{leap}].
		
		\begin{figure}[H]
			\centering
			\includegraphics[scale=0.55]{"Immagini workbook/Immagini esp3/leap"}
			\caption{\textit{Differenza tra UT, TAI e UTC e visualizzazione dei leap seconds.}}
			\label{leap}
		\end{figure}
		
		I sistemi GNSS adoperano un sistema di orologi atomici per scandire il tempo, sia a bordo dei satelliti ma anche nel segmento di terra, definendo il tempo di missione noto rispettivamente per la costellazione statunitense e per quella europea come \textit{GPS Time} (GPST) e \textit{Galileo Standard Time} (GST). Il punto di inizio di questi sistemi di tempo è rispettivamente il 6 Gennaio 1980 a 00:00UTC e il 22 Agosto 1999 a 00:00UTC, ma essendo il tempo scandito da orologi atomici, è necessario periodicamente "aggiustare" il tempo registrato dal GPST o dal GST con degli opportuni \textit{leap seconds}. Nel caso in esame, 18 secondi di \textit{leap seconds} erano stati registrati nel 2017. Infine, i satelliti delle costellazioni GNSS trasmettono nel messaggio di navigazione le informazioni necessarie per computare il tempo UTC corrente al momento dell'acquisizione del segnale.		
	\end{itemize}
	
	Successivamente, si procede alla lettura dei documenti RINEX ottenuti durante le acquisizioni, tramite la MATLAB \texttt{function} \texttt{rinexread.m} del \textit{Navigation Toolbox}. 
	
	Più osservazioni sono state eseguite per determinare la posizione di interesse e per ogni osservazione c'erano più satelliti in vista di più costellazioni. La costellazione con cui si esegue la misurazione delle posizioni è stata discriminata a inizio codice e rimane quindi computare lo pseudorange di ogni satellite della costellazione scelta. Si determina poi la posizione del ricevitore implementando un metodo dei minimi quadrati, che faccia riferimento ad una posizione iniziale (\texttt{RecPos0}) previamente consegnata (il documento di riferimento dipende dalla acquisizione fatta e ha estensione \textsl{.sbp.obs}), espressa nel sistema di riferimento ECEF. Nel caso in esame avrà forma:
	
	\begin{equation}
		\begin{aligned}
			\texttt{RecPos0} = \left.  \begin{bmatrix}
				X \\
				Y \\
				Z
			\end{bmatrix} \right|_{ECEF} \ \ [m]
		\end{aligned}
	\end{equation}

	Si procede quindi nel caricare le soluzioni fiduciarie di posizione, acquisite il 6 Dicembre 2022, elaborando i dati dal documento relativo documento con estensione \textsl{.mat}. Queste posizioni fiduciarie sono poi riportate in sovra-impressione su un immagine satellitare del luogo di interesse e non più espresse in coordinate ECEF, ma in coordinate geografiche: questo è possibile tramite l'uso del codice RTKLIB. Per passare da un sistema di coordinate all'altro, si usa la MATLAB \texttt{function ecef2lla}. %[fig. \ref{lla1}]
	
%	\begin{figure}[H]
%		\centering
%		\subfloat[\textit{Soluzioni fiduciarie visualizzate in sovra-impressione sull'immagine satellitare.}]{{\includegraphics[scale=0.37]{"Immagini workbook/Immagini esp3/fid1"}}}
%		\qquad
%		\subfloat[\textit{Particolare delle soluzioni fiduciarie.}]{{\includegraphics[scale=0.37]{"Immagini workbook/Immagini esp3/fid2"} }}
%		\caption{\textit{Visualizzazione delle soluzioni fiduciarie in coordinate geografiche.}}
%		\label{lla1}
%	\end{figure}

	Si procede quindi ad analizzare i \textit{Navigation Broadcast Messages} per ogni satellite in vista della costellazione considerata e successivamente si implementa un metodo dei minimi quadrati per la determinazione della posizione del ricevitore, a partire dalle misure di pseudorange ottenute dalle acquisizioni dei messaggi di navigazione trasmessi, racchiuse nei relativi documenti con estensione \textsl{.sbp.nav}. Quest'ultima parte è implementata nella specifica MATLAB \texttt{function GNSS$\_$fix$\_$ti.m}, che presenta anche tutti gli accorgimenti necessari per modellare la misurazione di pseudo-range $p^S_r$, affinché sia il meno affetta da errori.
	
	Le posizioni così ottenute dall'acquisizione dei messaggi trasmessi di navigazione vengono poste in sovra-impressione sulla mappa geografica del luogo dove è stata compiuta la misurazione e confrontate con le posizioni fiduciarie.
	
	\subsection{Correzioni alla misura di pseudorange tramite la \texttt{function GNSS$\_$fix$\_$ti.m}}
	 
	La misurazione di pseudo-range accorpa una serie di fattori, dove [fig. \ref{fig:pseudo1}]:
	
	\begin{equation}
		p^S_r = \rho^S_r + c (dt_r - dt^S) + T^S_r + I^S_r + K_r - K^S + e^S_r
	\end{equation}

	\begin{itemize}[-]
		\item $\rho^S_r = || \underline{r}^S - \underline{r}_r ||_2 $ è la norma euclidea, ovvero la distanza, tra il satellite e il ricevitore;
		
		\item $c (dt_r - dt^S)$ è l'errore di sincronizzazione tra gli orologi del satellite e del ricevitore;
		
		\item $T^S_r$ è il contributo di errore dato dall'attraversamento della troposfera;
		
		\item $I^s_r$ è il contributo di errore dato dall'attraversamento della ionosfera;
		
		\item $K^S$ e $K_r$ sono gli errori strumentali commessi rispettivamente dal satellite e dal ricevitore;
		
		\item $e^S_r$ è il contributo di errori di natura stocastica.
	\end{itemize}

	\begin{figure}[H]
		\centering
		\includegraphics[scale=0.6]{"Immagini workbook/Immagini esp3/pseudo1"}
		\caption{\textit{Elenco dei vari contributi che competono alla misura di pseudorange.}}
		\label{fig:pseudo1}
	\end{figure}
	

	Nella \texttt{function GNSS$\_$fix$\_$ti.m} si modellano e si correggono solo i contributi dovuti alla presenza della ionosfera, ad errori di sincronizzazione degli orologi, alla precisione delle coordinate dei satelliti e alla presenza della troposfera.
	
	Nel dettaglio:
	
	\begin{itemize}[-]
		
		\item Modello correttivo dello pseudorange a causa dell'interazione con la ionosfera:
		
		La ionosfera è un mezzo dispersivo per segnali modulati con portanti a radiofrequenze tipiche dei segnali GNSS come L1 e L2. Per compensare gli errori dovuti ai ritardi ionosferici nella propagazione del segnale, si possono combinare le misure di pseudorange nel caso in cui si disponga di ricevitori a doppi-frequenza, come quello in analisi. Lo pseudorange al netto dei ritardi ionosferici $p_{iono-free}$ si computa come combinazione lineare dei pseudorange misurati con il segnale di ciascuna singola portante ($p_{L1}$ e $p_{L2}$):
		
		\begin{equation}
			p_{iono-free} = a_1 \cdot p_{L1} + a_2 \cdot p_{L2} 
		\end{equation}
	
		dove gli $p_{L1}$ e $p_{L2}$ sono ottenuti dalla lettura dei documenti di osservazione (estensione \textsl{.sbp.obs}) e i coefficienti lineari sono funzioni delle frequenze delle portanti stesse (che sono parametri caratteristici di ogni costellazione):
		
		\begin{equation}
			a_1 = \frac{f^2_{L1}}{f^2_{L1} - f^2_{L2}} \ \ , \ \ a_2 = 1 - a_1 
		\end{equation}
	
		\item Modello correttivo del tempo registrato a causa dell'errore di sincronizzazione degli orologi:
		
		L'errore di sincronizzazione che si commette nella scala di tempo nel sistema satellite-ricevitore è composto da due contributi:
		
		\begin{enumerate}
			\item Errore di sincronizzazione dell'orologio del ricevitore $dt_r$, che viene stimato nel computo delle coordinate del ricevitore stesso;
			
			\item Errore di sincronizzazione dell'orologio di bordo del satellite $dt^S$, che subisce delle modifiche a causa di:
			
			\begin{enumerate}
				\item drift e bias commessi nella misurazione del tempo dagli orologi atomici di bordo, accorpati nel termine ($\tilde{dt}^S$). Questo termine è computabile tramite un polinomio di secondo grado, i cui coefficienti $a_i$ ($i=0,1,2$) sono trasmessi nel messaggio di navigazione:
				
				\begin{equation}
					\tilde{dt}^S = a_0 + a_1 \Delta t + a_2 \Delta t^2
				\end{equation}
			
				dove $\Delta t$ è la differenza di tempo tra il momento di acquisizione del messaggio di navigazione e di trasmissione del messaggio stesso.
				
				\item effetti relativistici dovuti alla variazione dell'eccentricità orbitale, accorpati nel termine $dt^S_{rel}$. Questo contributo è stimato come:
				
				\begin{equation}
					dt^S_{rel} = -2 \frac{\sqrt{\mu a}}{c^2} e \sin E
				\end{equation}
			
				dove $a$ ed $e$ sono rispettivamente il semiasse maggiore e l'eccentricità dell'orbita al tempo di emissione del messaggio di navigazione e le informazioni relative ai due parametri sono estratti dal messaggio di navigazione stesso; $E$ è l'anomalia eccentrica dell'orbita, sempre al momento dell'emissione del messaggio di navigazione, calcolata con l'equazione di Keplero.
			\end{enumerate}
			
		\end{enumerate} 
	
		\item Modello correttivo per le coordinate dei satelliti a causa della rotazione terrestre:
	
		Le \textit{broadcast ephemeresis} impiegano un certo tempo per raggiungere il ricevitore, il cosiddetto "tempo di volo" (in inglese, \textit{time of flight}). Dunque, il tempo di emissione dal satellite e di ricezione al ricevitore del messaggio di navigazione non è lo stesso. Durante il tempo di volo del segnale, il ricevitore si è spostato poichè ruota insieme alla Terra [fig. \ref{rot1}]. Si rivela necessario quindi considerare la rotazione terrestre durante il tempo di volo e computare così le coordinate del satellite non al tempo di emissione del segnale di navigazione, bensì al tempo di acquisizione del segnale stesso al ricevitore.
		
		\begin{figure}[H]
			\centering
			\includegraphics[scale=0.4]{"Immagini workbook/Immagini esp3/rot1"}
			\caption{\textit{Giustificazione grafica del considerare le coordinate del satellite al tempo di acquisizione, piuttosto che a quello di trasmissione del segnale di navigazione.}}
			\label{rot1}
		\end{figure}
		
		Questo è possibile esprimendo le coordinate del satellite in un sistema ECEF ruotato di un angolo pari all'angolo di rotazione terrestre durante il tempo di volo del segnale, attorno all'asse Z:
		
		\begin{equation}
			\left. \underline{r}^S\right|_{time \ of \ reception} = [R_3(\omega_E \Delta t)] \cdot \left. \underline{\tilde{r}}^S\right|_{time \ of \ transmission}
		\end{equation}
	
		\item Modello correttivo del pseudorange dovuto all'interazione con la troposfera:
		
		Essendo la troposfera un mezzo non dispersivo per i radiosegnali in frequenze tipiche dei sistemi GNSS, l'errore troposferico non può essere eliminato con un misure a doppia frequenza, come per la correzione di errori ionosferici. Si rende necessario quindi implementare modelli che stimino il meglio possibile il contributo degli errori troposferici.
		
		In questo studio si è implementato il modello di Saastamoinen, che fa riferimento ad un modello semplificato dell'atmosfera. In questo modello in funzione della:
		
		\begin{enumerate}
			\item lunghezza d'onda $\lambda$ del segnale GPS espresso in metri: $\lambda = 0,077m$;
			
			\item temperatura $T$ registrata il giorno dell'acquisizione in Kelvin: $T = 290,15K$;
			
			\item pressione atmosferica $P$ registrata il giorno dell'acquisizione in milliBar: $P = 1017mBar$;
			
			\item pressione parziale $e$ del vapore acqueo nella troposfera in millibar: $e = 6,11mBar$.
			
			\item altezza $h$ del ricevitore in metri.
		\end{enumerate}
	
		Il contributo del errore troposferico in metri si computa quindi come:
		
		\begin{equation}
			Trop_{correction} = \left[  \frac{2,277 \cdot P}{T} \cdot \left( 1 +  \frac{1'255}{T} + \frac{0,05}{h} \right) \cdot \frac{1 - e}{P}\right] \cdot \lambda \ \ [m]
		\end{equation}
	\end{itemize}

	Dopo aver modellato tutti gli errori commessi nella misurazione dello pseudorange per ogni satellite della costellazione considerata, si implementa un metodo dei minimi quadrati per la stima della posizione del ricevitore per ogni osservazione effettuata, con rispettiva analisi di covarianza. La posizione del ricevitore è espressa così in un sistema di riferimento ECEF, ma con le opportune trasformazioni di coordinate, si possono esprimere in un sistema di riferimento geodetico o ENU. In modo analogo, le matrici di covarianza devono essere moltiplicate per lo jacobiano e il suo trasposto necessario per effettuare il cambio di sistema di riferimento.
	
	La matrice di covarianza ottenuta è una matrice 4x4. Nel caso di una soluzione espressa in un sistema di riferimento ENU, i primi tre elementi della diagonale sono i quadrati delle varianze delle misurazioni di posizione nelle direzioni del sistema di riferimento ENU e il quarto elemento della diagonale è il quadrato della varianza della misurazione sul \textit{clock offset}.
	
	Le incertezze su questi quattro parametri che definiscono la posizione finale stimata del ricevitore si ottengono riscalando i quadrati delle varianze per un opportuno fattore di copertura: essendo un caso tridimensionale, affinché si abbia un livello di confidenza del 99,5$\%$ è necessario porre un fattore di copertura di 3,7.
	
	Ottenute le incertezze per ogni parametro, è possibile computare vari parametri di \textit{Dilution Of Precision}, come:
	
	\begin{itemize}[-]
		\item \textit{Vertical Dilution Of Precision} = VDOP = $\sigma_u$;
		
		\item \textit{Horizontal Dilution Of Precision} = VDOP = $\sqrt{\sigma_n^2 + \sigma_e^2}$;
		
		\item \textit{Position Dilution Of Precision} = PDOP = $\sqrt{\sigma_n^2 + \sigma_e^2 + \sigma_u^2}$;
		
		\item \textit{Time Dilution Of Precision} = TDOP = $\sigma_{\tau}^2$;
		
		\item \textit{Geometric Dilution Of Precision} = GDOP = $\sqrt{\sigma_n^2 + \sigma_e^2 + \sigma_u^2 + \sigma_{\tau}^2}$;
		
	\end{itemize}
	
	\section{Allestimento Sperimentale}
	
	Il setup sperimentale è composto da:
	
	\begin{itemize}[-]
		\item Un ricevitore GNSS professionale della \textit{Swift Navigation}, modello "Piksi Multi GNSS Module", che acquisisce dati in multi-frequenza, da più costellazioni GNSS.;
		
		\item Una "Swift Console" per la registrazione dei dati raccolti e la loro visualizzazione;
		
		\item Un modulo per la conversione dei file da formato binario SBP (\textit{Swif Binary Protocol}) a RINEX;
		
		\item Un computer con installato MATLAB (R2022b o versioni successive), con annesso \textit{Navigation Toolbox}.
		
		\end{itemize}
	
	Si sceglie anche una località all'aperto, il più distante da sorgenti di \textit{multipath} possibili, al fine di acquisire i dati. Nel caso di studio si è scelta la zona universitaria del Portello di Padova.
	
	I dati possono essere acquisiti staticamente, ma anche in maniera dinamica, spostandosi a passo d'uomo. Sono state scelte due posizioni statiche: una di inizio, presso il Complesso di Ingegneria Meccanica dell'Università di Padova, e una di fine, presso il Complesso di Geoscienze dell'Università degli Studi di Padova. Le acquisizioni dinamiche sono state effettuate effettuando il percorso pedonale che congiunge i punti di inizio e di fine, attraversando il Ponte del Portello.
	
	\section{Risultati}
	
	I risultati dipendono se l'acquisizione è stata fatta in modo statico o dinamico. Si differenziano tre casi: acquisizione statica di partenza, acquisizione dinamica, acquisizione statica di arrivo. Inoltre, la costellazione scelta per le analisi è quella statunitense GPS.
	
	\subsection{Acquisizione statica di partenza}
	
	I risultati ottenuti dall'acquisizione statica di partenza sono esplicate al meglio dai seguenti grafici e mappature [fig. \ref{fig:stat12}, \ref{fig:stat13}, \ref{fig:stat15}].
	
	\begin{figure}[H]
		\centering
		\includegraphics[scale=0.30]{"Immagini workbook/Immagini esp3/stat12"}
		\caption{\textit{Visualizzazione delle posizioni stimate (blu) e posizioni fiduciarie di partenza (rosse).}}
		\label{fig:stat12}
	\end{figure}

	\vspace{-0.5cm}

	\begin{figure}[H]
		\centering
		\includegraphics[scale=0.25]{"Immagini workbook/Immagini esp3/stat13"}
		\caption{\textit{Variazione delle varie formulazioni per il DOP lungo il tempo di acquisizione per il caso statico di partenza.}}
		\label{fig:stat13}
	\end{figure}

	\vspace{-0.5cm}

		%\begin{figure}[H]
	%	\centering
	%	\includegraphics[scale=0.25]{"Immagini workbook/Immagini esp3/stat14"}
	%	\caption{\textit{Numero dei satelliti GPS in vista in funzione del tempo di acquisizione per il caso statico di partenza.}}
	%	\label{fig:stat14}
	%\end{figure}

		\begin{figure}[H]
		\centering
		\includegraphics[scale=0.25]{"Immagini workbook/Immagini esp3/stat15"}
		\caption{\textit{Skyplot dei satelliti GPS in vista al momento dell'acquisizione statica di partenza.}}
		\label{fig:stat15}
	\end{figure}

	\subsection{Acquisizione dinamica}
	
	I risultati ottenuti dalla acquisizioni dinamica sono esplicate al meglio dai seguenti grafici e mappature [fig. \ref{fig:din2}, \ref{fig:din3}, \ref{fig:din5}].
	
	\begin{figure}[H]
		\centering
		\includegraphics[scale=0.30]{"Immagini workbook/Immagini esp3/din2"}
		\caption{\textit{Visualizzazione delle posizioni stimate (blu) e posizioni fiduciarie dinamiche (rosse).}}
		\label{fig:din2}
	\end{figure}
	
	\vspace{-0.5cm}
	
	\begin{figure}[H]
		\centering
		\includegraphics[scale=0.25]{"Immagini workbook/Immagini esp3/din3"}
		\caption{\textit{Variazione delle varie formulazioni per il DOP lungo il tempo di acquisizione per il caso dinamico.}}
		\label{fig:din3}
	\end{figure}
	
	\vspace{-0.5cm}
	
	%\begin{figure}[H]
	%	\centering
	%	\includegraphics[scale=0.25]{"Immagini workbook/Immagini esp3/din4"}
	%	\caption{\textit{Numero dei satelliti GPS in vista in funzione del tempo di acquisizione per il caso dinamico.}}
	%	\label{fig:din4}
	%\end{figure}
	
	\begin{figure}[H]
		\centering
		\includegraphics[scale=0.25]{"Immagini workbook/Immagini esp3/din5"}
		\caption{\textit{Skyplot dei satelliti GPS in vista al momento dell'acquisizione dinamica.}}
		\label{fig:din5}
	\end{figure}

	\subsection{Acquisizione statica di arrivo}
	
	I risultati ottenuti dall'acquisizione statica di arrivo sono esplicate al meglio dai seguenti grafici e mappature [fig. \ref{fig:stat22}, \ref{fig:stat23}, \ref{fig:stat25}].
	
	\begin{figure}[H]
		\centering
		\includegraphics[scale=0.30]{"Immagini workbook/Immagini esp3/stat22"}
		\caption{\textit{Visualizzazione delle posizioni stimate (blu) e posizioni fiduciarie di arrivo (rosse).}}
		\label{fig:stat22}
	\end{figure}
	
	\vspace{-0.5cm}
	
	\begin{figure}[H]
		\centering
		\includegraphics[scale=0.25]{"Immagini workbook/Immagini esp3/stat23"}
		\caption{\textit{Variazione delle varie formulazioni per il DOP lungo il tempo di acquisizione per il caso statico di arrivo.}}
		\label{fig:stat23}
	\end{figure}
	
	\vspace{-0.5cm}
	
	%\begin{figure}[H]
	%	\centering
	%	\includegraphics[scale=0.25]{"Immagini workbook/Immagini esp3/stat24"}
	%	\caption{\textit{Numero dei satelliti GPS in vista in funzione del tempo di acquisizione per il caso statico di arrivo.}}
	%	\label{fig:stat24}
	%\end{figure}
	
	\begin{figure}[H]
		\centering
		\includegraphics[scale=0.25]{"Immagini workbook/Immagini esp3/stat25"}
		\caption{\textit{Skyplot dei satelliti GPS in vista al momento dell'acquisizione statica di arrivo.}}
		\label{fig:stat25}
	\end{figure}

	\section{Conclusioni}
	
	Si osserva come le soluzioni di posizione ottenute dai dati acquisiti si scostano dalle soluzioni fiduciarie di posizione dell'ordine del metro e dei suoi sottomultipli, come ci si può attendere dalla bontà del ricevitore adottato e dal modello delle correzioni usate nel computo dei pseudorange.
	
	Si osserva nel dettaglio [tab. \ref{taberr3}] come l'assenza di una singola correzione vari l'accuratezza della posizione iniziale stimata del ricevitore:
	
	\begin{table}[H]
		\centering
		\begin{tabular}{|M{70mm}|M{40mm}|M{40mm}|}
			\hline
			\textbf{Errore non corretto} & \textbf{Errore lungo la direzione Nord [m]} & \textbf{Errore lungo la direzione Est [m]} \\
			\hline
			Errore di sincronizzazione degli orologi a bordo dei satelliti & +29'170 & -66'670 \\
			\hline
			Errore di mancata considerazione dei \textit{leap seconds} & -4'180 & +2'000 \\
			\hline
			Errore sulla posizione dei satelliti & -19,77 & +6,82 \\
			\hline
			Errore di Segnac & -1,79 & +5 \\
			\hline
			Errore relativistico & -3,21 & +2,07 \\
			\hline
			Errore ionosferico & -4,70  & -1,23 \\
			\hline
			Errore troposferico & -1,10 & 1,83 \\
			\hline \hline
			\textbf{ERRORE FINALE COMMESSO} & \textbf{-1,91}  & \textbf{-1,13} \\
			\hline
		\end{tabular}
	\caption{\textit{Contributo dei singoli errori nel computo dello pseudorange per la posizioni iniziale di partenza.}}\label{taberr3}
	\end{table}

	In modo analogo si possono computare i singoli contributi degli errori nella misurazione dello pseudorange anche per i dati raccolti durante l'acquisizione dinamica e quella statica di arrivo.
	
	In appendice [cfr. Appendice, par. 5.1] si trovano le immagini di come la posizione iniziale del ricevitore stimata si scosti dalla posizione fiduciaria in funzione della correzione o meno di un singolo errore nel computo dello pseudorange del ricevitore dal satellite della costellazione GPS.
	
	Si pone ora particolare attenzione ai vari valori di DOP rilevati dalle acquisizioni:
	
	\begin{itemize}[-]
		\item Acquisizione statica di partenza:
		
		In prossimità del 500esimo secondo a partire dall'inizio del tempo di acquisizione dei dati, si osserva per un bravissimo istante un errore nella lettura dei satelliti in vista, che passano da 7 a 2, venendo a meno il numero di satelliti minimi necessari per la quadrilaterazione, ovvero 4 satelliti. Questa mancata acquisizione di satelliti si traduce in un picco in tutti i vari parametri di DOP. Tuttavia, nella maggior parte del tempo speso per l'acquisizione dei dati nel caso statico di partenza si hanno in vista 7 satelliti. Eventuali variazioni del numero dei satelliti, specie se diminuiscono, si traducono in peggioramenti dei valori di DOP. Infine, mediamente il valore di DOP è inferiore a 5 e avendo incertezze dell'ordine della decina di centimetro sulle direzioni lungo la terna ENU, giustifica l'errore finale complessivo sulla misura di posizione dell'ordine dell'unità del metro.
		
		Qui si hanno i valori di DOP maggiori rispetto a quelli ottenuti negli altri casi di acquisizione (ovvero il caso dinamico e il caso statico di arrivo), poiché come si osserva nel relativo \textit{skyplot} [fig. \ref{fig:stat15}], i satelliti sono disposti lungo direzioni preferenziali sia lungo la direzione Nord, che quella Est.
		
		\item Acquisizione dinamica:
		
		Le considerazioni sono analoghe a quelle menzionate per il caso precedente di acquisizione statica di partenza. 
		
		Tuttavia, è interessante notare come intorno al 150esimo secondo a partire dall'inizio dell'acquisizione dei dati si osservano delle discontinuità nel tracciamento dei valori di DOP. Ciò è dovuto all'attraversamento di una zona coperta, ovvero la Porta Ognissanti presente sul Ponte del Portello, che scherma il segnale tramesso dai satelliti di una qualsiasi costellazione GNSS (in questo caso, GPS). Il riscontro è anche visivo in [fig. \ref{fig:din2}], dove mancano le posizioni fiduciarie e stimate del ricevitore.
		
		\item Acquisizione statica di arrivo:
		
		Qui si osservano i valori più bassi di DOP: questo è dovuto al fatto che nel tempo voluto per arrivare alla posizione statica di arrivo da quella di partenza, i satelliti della costellazione GPS si sono dispersi maggiormente. Ancora una volta, si osserva come il valore di DOP sia sensibile al numero di satelliti in vista. 
	\end{itemize}

	In generale, l'andamento frastagliato dei valori valori di DOP durante il tempo di acquisizione è dovuto alla variazione del numero di satelliti in vista. Al diminuire del numero di satelliti in vista e all'aumentare della loro disposizione lungo direzioni preferenziali, i valori di DOP aumentano, decretando una diminuzione dell'accuratezza della posizione finale stimata.

	Si evince da questa esperienza l'importanza di modellare correttamente gli errori che incorrono all'atto del computo del pseudorange e di compiere acquisizioni possibilmente in spazi aperti, lontani da palazzi o sorgenti di \textit{multipath}, in momenti in cui si abbia il numero massimo di satelliti in vista e che la la dispersione nella volta celeste sia massima.

	\section{Bibliografia}
	
	[1] - Andrea Valmorbida, PhD, \textit{Appunti del Corso di "Satellite Navigation"}, Università degli Studi di Padova, Dipartimento di Ingegneria Industriale (dii), 2024.\\
	
	[2] - A. Martellucci and R. Prieto Cerdeira, \textit{Review of tropospheric, ionospheric and multipath data and models for Global Navigation Satellite Systems}, 3rd European Conference on Antennas and Propagation (EuCAP) in 2009, Berlin (Germany).
	

	\chapter{Ottava Esperienza di Laboratorio}
	
	\section{Abstract}
	
	L'obiettivo di questa esperienza è di ottenere la posizione del ricevitore usando misure di pseudorange, adoperando le informazioni ottenute sia dalle effemeridi trasmesse dai satelliti, sia da quelle ottenute dai prodotti precisi.
	
	\section{Introduzione}
	
	Manca
	
	\section{Metodo di Analisi}
	
	Il codice usato per ottenere la posizione finale del ricevitore è \texttt{GNSS$\_$SPP$\_$main.m}, nel quale è possibile analizzare solo una costellazione GNSS alla volta, specificando il suo nome all'inizio.
	
	Segue poi la definizioni delle costanti fisiche principali come:
	
	\begin{itemize}[-]
		\item la velocità della luce, pari a $c = 299792458 m/s$;
		
		\item la costante di gravitazione per la Terra, facente riferimento al modello geodetico di riferimento WGS-84, pari a $\mu = 3.986005 \cdot 10 ^{14}m^3/s^2$;
		
		\item il raggio terrestre medio, pari a $R_E = 6'378'137m$;
		
		\item la velocità angolare di rotazione terrestre media, pari a $\omega_E = 7.2921151467 \cdot 10^{-5} rad/s$, secondo i modelli usati dalle costellazioni GPS e Galileo;
		
		\item la costante di accelerazione gravitazionale terrestre, pari a $g = 9.80665m/s^2$;
		
		\item i \textit{leap seconds} pari a $18$ secondi. Prima di esplicare l'origine di questi, è bene  ricordare che esiste una discrepanza registrata tra il tempo scandito dal moto della Terra rispetto al Sole, noto come \textit{Universal Time} (UT), e quello scandito da un sistema di orologi atomici, gestiti da il \textit{Bureau International de l'Heure} (BIH) e noto come \textit{International Atomic Time} (TAI). Siccome la rotazione della Terra non è costante, il tempo scandito dal sistema UT non è uniforme e accumula un \textit{drift} rispetto al tempo registrato dal sistema TAI. Tuttavia, si definisce lo \textit{Corrdinated Universal Time} (UTC), che è un compromesso tra il tempo registrato dal sistema TAI e dal sistema UT. Lo UTC segue il tempo scandito dal sistema TAI, a cui viene "sottratto" periodicamente un certo numero di secondi, detti appunto \textit{leap seconds}, in modo che il tempo registrato abbia una discrepanza al massimo di 0,9 secondi rispetto a quello segnato dal sistema UT [fig. \ref{leap}].
		
		%\begin{figure}[H]
		%	\centering
		%	\includegraphics[scale=0.55]{"Immagini workbook/Immagini esp3/leap"}
		%	\caption{\textit{Differenza tra UT, TAI e UTC e visualizzazione dei leap seconds.}}
		%	\label{leap}
		%\end{figure}
		
		I sistemi GNSS adoperano un sistema di orologi atomici per scandire il tempo, sia a bordo dei satelliti ma anche nel segmento di terra, definendo il tempo di missione noto rispettivamente per la costellazione statunitense e per quella europea come \textit{GPS Time} (GPST) e \textit{Galileo Standard Time} (GST). Il punto di inizio di questi sistemi di tempo è rispettivamente il 6 Gennaio 1980 a 00:00UTC e il 22 Agosto 1999 a 00:00UTC, ma essendo il tempo scandito da orologi atomici, è necessario periodicamente "aggiustare" il tempo registrato dal GPST o dal GST con degli opportuni \textit{leap seconds}. Nel caso in esame, 18 secondi di \textit{leap seconds} erano stati registrati nel 2017. Infine, i satelliti delle costellazioni GNSS trasmettono nel messaggio di navigazione le informazioni necessarie per computare il tempo UTC corrente al momento dell'acquisizione del segnale.		
	\end{itemize}
	
	Successivamente, si procede alla lettura dei documenti RINEX ottenuti durante le acquisizioni, tramite la MATLAB \texttt{function} \texttt{rinexread.m} del \textit{Navigation Toolbox}. 
	
	Più osservazioni sono state eseguite per determinare la posizione di interesse e per ogni osservazione c'erano più satelliti in vista di più costellazioni. La costellazione con cui si esegue la misurazione delle posizioni è stata discriminata a inizio codice e rimane quindi computare lo pseudorange di ogni satellite della costellazione scelta. Si determina poi la posizione del ricevitore implementando un metodo dei minimi quadrati, che faccia riferimento ad una posizione iniziale (\texttt{RecPos0}) previamente consegnata (il documento di riferimento dipende dalla acquisizione fatta e ha estensione \textsl{.sbp.obs}), espressa nel sistema di riferimento ECEF. Nel caso in esame avrà forma:
	
	\begin{equation}
		\begin{aligned}
			\texttt{RecPos0} = \left.  \begin{bmatrix}
				X \\
				Y \\
				Z
			\end{bmatrix} \right|_{ECEF} \ \ [m]
		\end{aligned}
	\end{equation}
	
	Si procede quindi nel caricare le soluzioni fiduciarie di posizione, acquisite il 29 Maggio 2024, elaborando i dati dal documento relativo con estensione \textsl{.mat}. Queste posizioni fiduciarie sono poi riportate in sovra-impressione su un immagine satellitare del luogo di interesse e non più espresse in coordinate ECEF, ma in coordinate geografiche: questo è possibile tramite l'uso del codice RTKLIB. Per passare da un sistema di coordinate all'altro, si usa la MATLAB \texttt{function ecef2lla}. %[fig. \ref{lla1}]
	
	%	\begin{figure}[H]
		%		\centering
		%		\subfloat[\textit{Soluzioni fiduciarie visualizzate in sovra-impressione sull'immagine satellitare.}]{{\includegraphics[scale=0.37]{"Immagini workbook/Immagini esp3/fid1"}}}
		%		\qquad
		%		\subfloat[\textit{Particolare delle soluzioni fiduciarie.}]{{\includegraphics[scale=0.37]{"Immagini workbook/Immagini esp3/fid2"} }}
		%		\caption{\textit{Visualizzazione delle soluzioni fiduciarie in coordinate geografiche.}}
		%		\label{lla1}
		%	\end{figure}
	
	Si procede quindi ad analizzare i \textit{Navigation Broadcast Messages} per ogni satellite in vista della costellazione considerata e successivamente si implementa un metodo dei minimi quadrati per la determinazione della posizione del ricevitore, a partire dalle misure di pseudorange ottenute dalle acquisizioni dei messaggi di navigazione trasmessi, racchiuse nei relativi documenti con estensione \textsl{.sbp.nav}. Quest'ultima parte è implementata nella specifica MATLAB \texttt{function GNSS$\_$fix$\_$ti.m}, che presenta anche tutti gli accorgimenti necessari per modellare la misurazione di pseudo-range $p^S_r$, affinché sia il meno affetta da errori.
	
	Le posizioni così ottenute dall'acquisizione dei messaggi trasmessi di navigazione vengono poste in sovra-impressione sulla mappa geografica del luogo dove è stata compiuta la misurazione e confrontate con le posizioni fiduciarie.
	
	\subsection{Correzioni alla misura di pseudorange tramite la \texttt{function GNSS$\_$fix$\_$ti.m}}
	
	La misurazione di pseudo-range accorpa una serie di fattori, dove [fig. \ref{fig:pseudo1}]:
	
	\begin{equation}
		p^S_r = \rho^S_r + c (dt_r - dt^S) + T^S_r + I^S_r + K_r - K^S + e^S_r
	\end{equation}
	
	\begin{itemize}[-]
		\item $\rho^S_r = || \underline{r}^S - \underline{r}_r ||_2 $ è la norma euclidea, ovvero la distanza, tra il satellite e il ricevitore;
		
		\item $c (dt_r - dt^S)$ è l'errore di sincronizzazione tra gli orologi del satellite e del ricevitore;
		
		\item $T^S_r$ è il contributo di errore dato dall'attraversamento della troposfera;
		
		\item $I^s_r$ è il contributo di errore dato dall'attraversamento della ionosfera;
		
		\item $K^S$ e $K_r$ sono gli errori strumentali commessi rispettivamente dal satellite e dal ricevitore;
		
		\item $e^S_r$ è il contributo di errori di natura stocastica.
	\end{itemize}
	
	%\begin{figure}[H]
	%	\centering
	%	\includegraphics[scale=0.6]{"Immagini workbook/Immagini esp3/pseudo1"}
	%	\caption{\textit{Elenco dei vari contributi che competono alla misura di pseudorange.}}
	%	\label{fig:pseudo1}
	%\end{figure}
	
	
	Nella \texttt{function GNSS$\_$fix$\_$ti.m} si modellano e si correggono solo i contributi dovuti alla presenza della ionosfera, ad errori di sincronizzazione degli orologi, alla precisione delle coordinate dei satelliti e alla presenza della troposfera.
	
	Nel dettaglio:
	
	\begin{itemize}[-]
		
		\item Modello correttivo dello pseudorange a causa dell'interazione con la ionosfera:
		
		La ionosfera è un mezzo dispersivo per segnali modulati con portanti a radiofrequenze tipiche dei segnali GNSS come L1 e L2. Per compensare gli errori dovuti ai ritardi ionosferici nella propagazione del segnale, si possono combinare le misure di pseudorange nel caso in cui si disponga di ricevitori a doppi-frequenza, come quello in analisi. Lo pseudorange al netto dei ritardi ionosferici $p_{iono-free}$ si computa come combinazione lineare dei pseudorange misurati con il segnale di ciascuna singola portante ($p_{L1}$ e $p_{L2}$):
		
		\begin{equation}
			p_{iono-free} = a_1 \cdot p_{L1} + a_2 \cdot p_{L2} 
		\end{equation}
		
		dove gli $p_{L1}$ e $p_{L2}$ sono ottenuti dalla lettura dei documenti di osservazione (estensione \textsl{.sbp.obs}) e i coefficienti lineari sono funzioni delle frequenze delle portanti stesse (che sono parametri caratteristici di ogni costellazione):
		
		\begin{equation}
			a_1 = \frac{f^2_{L1}}{f^2_{L1} - f^2_{L2}} \ \ , \ \ a_2 = 1 - a_1 
		\end{equation}
		
		\item Modello correttivo del tempo registrato a causa dell'errore di sincronizzazione degli orologi:
		
		L'errore di sincronizzazione che si commette nella scala di tempo nel sistema satellite-ricevitore è composto da due contributi:
		
		\begin{enumerate}
			\item Errore di sincronizzazione dell'orologio del ricevitore $dt_r$, che viene stimato nel computo delle coordinate del ricevitore stesso;
			
			\item Errore di sincronizzazione dell'orologio di bordo del satellite $dt^S$, che subisce delle modifiche a causa di:
			
			\begin{enumerate}
				\item drift e bias commessi nella misurazione del tempo dagli orologi atomici di bordo, accorpati nel termine ($\tilde{dt}^S$). Questo termine è computabile tramite un polinomio di secondo grado, i cui coefficienti $a_i$ ($i=0,1,2$) sono trasmessi nel messaggio di navigazione:
				
				\begin{equation}
					\tilde{dt}^S = a_0 + a_1 \Delta t + a_2 \Delta t^2
				\end{equation}
				
				dove $\Delta t$ è la differenza di tempo tra il momento di acquisizione del messaggio di navigazione e di trasmissione del messaggio stesso.
				
				\item effetti relativistici dovuti alla variazione dell'eccentricità orbitale, accorpati nel termine $dt^S_{rel}$. Questo contributo è stimato come:
				
				\begin{equation}
					dt^S_{rel} = -2 \frac{\sqrt{\mu a}}{c^2} e \sin E
				\end{equation}
				
				dove $a$ ed $e$ sono rispettivamente il semiasse maggiore e l'eccentricità dell'orbita al tempo di emissione del messaggio di navigazione e le informazioni relative ai due parametri sono estratti dal messaggio di navigazione stesso; $E$ è l'anomalia eccentrica dell'orbita, sempre al momento dell'emissione del messaggio di navigazione, calcolata con l'equazione di Keplero.
			\end{enumerate}
			
		\end{enumerate} 
		
		\item Modello correttivo per le coordinate dei satelliti a causa della rotazione terrestre:
		
		Le \textit{broadcast ephemeresis} impiegano un certo tempo per raggiungere il ricevitore, il cosiddetto "tempo di volo" (in inglese, \textit{time of flight}). Dunque, il tempo di emissione dal satellite e di ricezione al ricevitore del messaggio di navigazione non è lo stesso. Durante il tempo di volo del segnale, il ricevitore si è spostato poichè ruota insieme alla Terra [fig. \ref{rot1}]. Si rivela necessario quindi considerare la rotazione terrestre durante il tempo di volo e computare così le coordinate del satellite non al tempo di emissione del segnale di navigazione, bensì al tempo di acquisizione del segnale stesso al ricevitore.
		
		%\begin{figure}[H]
		%	\centering
		%	\includegraphics[scale=0.4]{"Immagini workbook/Immagini esp3/rot1"}
		%	\caption{\textit{Giustificazione grafica del considerare le coordinate del satellite al tempo di acquisizione, piuttosto che a quello di trasmissione del segnale di navigazione.}}
		%	\label{rot1}
	%	\end{figure}
		
		Questo è possibile esprimendo le coordinate del satellite in un sistema ECEF ruotato di un angolo pari all'angolo di rotazione terrestre durante il tempo di volo del segnale, attorno all'asse Z:
		
		\begin{equation}
			\left. \underline{r}^S\right|_{time \ of \ reception} = [R_3(\omega_E \Delta t)] \cdot \left. \underline{\tilde{r}}^S\right|_{time \ of \ transmission}
		\end{equation}
		
		\item Modello correttivo del pseudorange dovuto all'interazione con la troposfera:
		
		Essendo la troposfera un mezzo non dispersivo per i radiosegnali in frequenze tipiche dei sistemi GNSS, l'errore troposferico non può essere eliminato con un misure a doppia frequenza, come per la correzione di errori ionosferici. Si rende necessario quindi implementare modelli che stimino il meglio possibile il contributo degli errori troposferici.
		
		In questo studio si è implementato il modello di Saastamoinen, che fa riferimento ad un modello semplificato dell'atmosfera. In questo modello in funzione della:
		
		\begin{enumerate}
			\item lunghezza d'onda $\lambda$ del segnale GPS espresso in metri: $\lambda = 0,077m$;
			
			\item temperatura $T$ registrata il giorno dell'acquisizione in Kelvin: $T = 290,15K$;
			
			\item pressione atmosferica $P$ registrata il giorno dell'acquisizione in milliBar: $P = 1017mBar$;
			
			\item pressione parziale $e$ del vapore acqueo nella troposfera in millibar: $e = 6,11mBar$.
			
			\item altezza $h$ del ricevitore in metri.
		\end{enumerate}
		
		Il contributo del errore troposferico in metri si computa quindi come:
		
		\begin{equation}
			Trop_{correction} = \left[  \frac{2,277 \cdot P}{T} \cdot \left( 1 +  \frac{1'255}{T} + \frac{0,05}{h} \right) \cdot \frac{1 - e}{P}\right] \cdot \lambda \ \ [m]
		\end{equation}
	\end{itemize}
	
	Dopo aver modellato tutti gli errori commessi nella misurazione dello pseudorange per ogni satellite della costellazione considerata, si implementa un metodo dei minimi quadrati per la stima della posizione del ricevitore per ogni osservazione effettuata, con rispettiva analisi di covarianza. La posizione del ricevitore è espressa così in un sistema di riferimento ECEF, ma con le opportune trasformazioni di coordinate, si possono esprimere in un sistema di riferimento geodetico o ENU. In modo analogo, le matrici di covarianza devono essere moltiplicate per lo jacobiano e il suo trasposto necessario per effettuare il cambio di sistema di riferimento.
	
	La matrice di covarianza ottenuta è una matrice 4x4. Nel caso di una soluzione espressa in un sistema di riferimento ENU, i primi tre elementi della diagonale sono i quadrati delle varianze delle misurazioni di posizione nelle direzioni del sistema di riferimento ENU e il quarto elemento della diagonale è il quadrato della varianza della misurazione sul \textit{clock offset}.
	
	Le incertezze su questi quattro parametri che definiscono la posizione finale stimata del ricevitore si ottengono riscalando i quadrati delle varianze per un opportuno fattore di copertura: essendo un caso tridimensionale, affinché si abbia un livello di confidenza del 99,5$\%$ è necessario porre un fattore di copertura di 3,7.
	
	Ottenute le incertezze per ogni parametro, è possibile computare vari parametri di \textit{Dilution Of Precision}, come:
	
	\begin{itemize}[-]
		\item \textit{Vertical Dilution Of Precision} = VDOP = $\sigma_u$;
		
		\item \textit{Horizontal Dilution Of Precision} = VDOP = $\sqrt{\sigma_n^2 + \sigma_e^2}$;
		
		\item \textit{Position Dilution Of Precision} = PDOP = $\sqrt{\sigma_n^2 + \sigma_e^2 + \sigma_u^2}$;
		
		\item \textit{Time Dilution Of Precision} = TDOP = $\sigma_{\tau}^2$;
		
		\item \textit{Geometric Dilution Of Precision} = GDOP = $\sqrt{\sigma_n^2 + \sigma_e^2 + \sigma_u^2 + \sigma_{\tau}^2}$;
		
	\end{itemize}
	
	\section{Allestimento Sperimentale}
	
	Il setup sperimentale è composto da:
	
	\begin{itemize}[-]
		\item Un ricevitore GNSS professionale della \textit{Swift Navigation}, modello "Piksi Multi GNSS Module", che acquisisce dati in multi-frequenza, da più costellazioni GNSS.;
		
		\item Una "Swift Console" per la registrazione dei dati raccolti e la loro visualizzazione;
		
		\item Un modulo per la conversione dei file da formato binario SBP (\textit{Swif Binary Protocol}) a RINEX;
		
		\item Un computer con installato MATLAB (R2022b o versioni successive), con annesso \textit{Navigation Toolbox}.
		
	\end{itemize}
	
	Si sceglie anche una località all'aperto, il più distante da sorgenti di \textit{multipath} possibili, al fine di acquisire i dati. Nel caso di studio si è scelto Parco d'Europa, anche noto come Parco della Musica, della città di Padova.
	
	I dati possono essere acquisiti staticamente, ma anche in maniera dinamica, spostandosi a passo d'uomo. Si è quindi effettuata una acquisizione statica in un luogo del Parco Europa il più all'aperto possibile, evitando la prossimità con palazzi e strutture, al fine di mitigare gli errori di \textit{multipath}. In un secondo momento, si è effettuata una acquisizione dinamica, seguendo il percorso pedonale del parco, compiendo un percorso chiuso.
	
	\section{Risultati}
	
	I risultati dipendono se l'acquisizione è stata fatta in modo statico o dinamico e nell'analisi dei risultati si differenziano i due casi. Inoltre, la costellazione scelta per le analisi è quella statunitense GPS.
	
	\subsection{Acquisizione statica}
	
	I risultati ottenuti dall'acquisizione statica sono esplicate al meglio dai seguenti grafici e mappature [fig. \ref{fig:stat821}, \ref{fig:stat822}, \ref{fig:stat83}, \ref{fig:stat85}].
	
	\begin{figure}[H]
		\centering
		\includegraphics[scale=0.30]{"Immagini workbook/Immagini esp8/stat821"}
		\caption{\textit{Vista satellitare del Parco d'Europa, Padova.}}
		\label{fig:stat821}
	\end{figure}
	
	\vspace{-0.5cm}
	
	\begin{figure}[H]
		\centering
		\includegraphics[scale=0.30]{"Immagini workbook/Immagini esp8/stat822"}
		\caption{\textit{Particolare sulle posizioni fiduciarie (blu) e stimate (rosse).}} 
		\label{fig:stat822}
	\end{figure}

	\begin{figure}[H]
		\centering
		\includegraphics[scale=0.30]{"Immagini workbook/Immagini esp8/stat83"}
		\caption{\textit{Particolare sulle posizioni fiduciarie (blu) e stimate (rosse).}} 
		\label{fig:stat83}
	\end{figure}
	
	\vspace{-0.5cm}
	
	%\begin{figure}[H]
	%	\centering
	%	\includegraphics[scale=0.25]{"Immagini workbook/Immagini esp3/stat14"}
	%	\caption{\textit{Numero dei satelliti GPS in vista in funzione del tempo di acquisizione per il caso statico di partenza.}}
	%	\label{fig:stat14}
	%\end{figure}
	
	\begin{figure}[H]
		\centering
		\includegraphics[scale=0.25]{"Immagini workbook/Immagini esp8/stat85"}
		\caption{\textit{Skyplot dei satelliti GPS in vista al momento dell'acquisizione statica di partenza.}}
		\label{fig:stat85}
	\end{figure}
	
	\subsection{Acquisizione dinamica}
	
	I risultati ottenuti dalla acquisizioni dinamica sono esplicate al meglio dai seguenti grafici e mappature [fig. \ref{fig:din82}, \ref{fig:din83}, \ref{fig:din85}].
	
	\begin{figure}[H]
		\centering
		\includegraphics[scale=0.30]{"Immagini workbook/Immagini esp8/din82"}
		\caption{\textit{Visualizzazione delle posizioni stimate (blu) e posizioni fiduciarie dinamiche (rosse).}}
		\label{fig:din82}
	\end{figure}
	
	\vspace{-0.5cm}
	
	\begin{figure}[H]
		\centering
		\includegraphics[scale=0.25]{"Immagini workbook/Immagini esp8/din83"}
		\caption{\textit{Variazione delle varie formulazioni per il DOP lungo il tempo di acquisizione per il caso dinamico.}}
		\label{fig:din83}
	\end{figure}
	
	\vspace{-0.5cm}
	
	%\begin{figure}[H]
	%	\centering
	%	\includegraphics[scale=0.25]{"Immagini workbook/Immagini esp3/din4"}
	%	\caption{\textit{Numero dei satelliti GPS in vista in funzione del tempo di acquisizione per il caso dinamico.}}
	%	\label{fig:din4}
	%\end{figure}
	
	\begin{figure}[H]
		\centering
		\includegraphics[scale=0.25]{"Immagini workbook/Immagini esp8/din5"}
		\caption{\textit{Skyplot dei satelliti GPS in vista al momento dell'acquisizione dinamica.}}
		\label{fig:din85}
	\end{figure}
	
	\section{Conclusioni}
	
	Si osserva come le soluzioni di posizione ottenute dai dati acquisiti si scostano dalle soluzioni fiduciarie di posizione dell'ordine del metro e dei suoi sottomultipli, come ci si può attendere dalla bontà del ricevitore adottato e dal modello delle correzioni usate nel computo dei pseudorange.
	
	Si osserva nel dettaglio [tab. \ref{taberr8}] come l'assenza di una singola correzione vari l'accuratezza della posizione stimata del ricevitore lungo il percorso dell'acquisizione dinamica, ma questo si può osservare analogamente anche per il caso dell'acquisizione statica:
	
	\begin{table}[H]
		\centering
		\begin{tabular}{|M{70mm}|M{40mm}|M{40mm}|}
			\hline
			\textbf{Errore non corretto} & \textbf{Errore lungo la direzione Nord [m]} & \textbf{Errore lungo la direzione Est [m]} \\
			\hline
			Errore di sincronizzazione degli orologi a bordo dei satelliti & +100'000 & -31'428 \\
			\hline
			Errore di mancata considerazione dei \textit{leap seconds} & -10'000 & +11'000 \\
			\hline
			Errore sulla posizione dei satelliti & -40 & +41,88 \\
			\hline
			Errore di Segnac & -0,25 & +23 \\
			\hline
			Errore relativistico & +0,92 & +9,21 \\
			\hline
			Errore ionosferico & +0,93  & +0,39 \\
			\hline
			Errore troposferico & -0,74 & -1,67 \\
			\hline \hline
			\textbf{ERRORE FINALE COMMESSO} & \textbf{-0,63}  & \textbf{-1,59} \\
			\hline
		\end{tabular}
		\caption{\textit{Contributo dei singoli errori nel computo dello pseudorange per la stima della posizione finale del riccevitore nel caso dell'acquisizione dinamica.}}\label{taberr8}
	\end{table}
	
	In appendice si trovano le immagini di come la posizione finale del ricevitore stimata si scosti dalla posizione fiduciaria in funzione della correzione o meno di un singolo errore nel computo dello pseudorange del ricevitore dal satellite della costellazione GPS.
	
	Si pone ora particolare attenzione ai vari valori di DOP rilevati dalle acquisizioni:
	
	\begin{itemize}[-]
		\item Acquisizione statica di partenza:
		
		Si osserva come la variazione dei vari valori di DOP lungo il tempo di acquisizione non sia frastagliata, bensì composta da una linea continua. Questo è dovuto al fatto che il numero dei satelliti in vista sia stato costante per tutto il tempo dell'acquisizione, pari a 8. Si osserva come tutti i vari valori di DOP diminuiscano nel tempo, mentre Il valore VDOP aumenti nel tempo, anche se di poco: questo è dovuto al fatto che la maggior parte dei satelliti in vista avesse un moto che puntava sullo Zenith dell'osservatore locale, come si osserva nel \textit{Skyplot} per i satelliti numero 8, 10, 21 e 27.
		
		\item Acquisizione dinamica:
		
		Si nota come il profilo dei vari valori del DOP sono molto frastagliati, in quanto si osservano numerose variazioni di satelliti in vista da cui si acquisisce il segnale, che varia da un minimo di 4 satelliti ad un massimo di 8.
		
		Si vede come in corrispondenza di un abbassamento del numero di satelliti in vista da cui si acquisisce il segnale, si riscontra un aumento di tutti i vari valori di DOP.
	\end{itemize}
	
	Si evince da questa esperienza l'importanza di modellare correttamente gli errori che incorrono all'atto del computo del pseudorange e di compiere acquisizioni possibilmente in spazi aperti, lontani da palazzi o sorgenti di \textit{multipath}, in momenti in cui si abbia la massima dispersione dei satelliti in vista.
	
	
	Si evince da questa esperienza l'importanza di modellare correttamente gli errori che incorrono all'atto del computo del pseudorange e di compiere acquisizioni possibilmente in spazi aperti, lontani da palazzi o sorgenti di \textit{multipath}, in momenti in cui si abbia il numero massimo di satelliti in vista e che la la dispersione nella volta celeste sia massima.
	
	\section{Bibliografia}
	
	[1] - Andrea Valmorbida, PhD, \textit{Appunti del Corso di "Satellite Navigation"}, Università degli Studi di Padova, Dipartimento di Ingegneria Industriale (dii), 2024.
	
	[2] - A. Martellucci and R. Prieto Cerdeira, \textit{Review of tropospheric, ionospheric and multipath data and models for Global Navigation Satellite Systems}, 3rd European Conference on Antennas and Propagation (EuCAP) in 2009, Berlin (Germany).
	
	\chapter{APPENDICE}
	
	\section{Appendice della Terza Esperienza di Laboratorio}

	\begin{figure}[H]
		\centering
		\includegraphics[scale=0.31]{"Immagini workbook/Immagini esp3/err_clock"}
		\caption{\textit{Errore commesso nella stima della posizione di partenza della terza esperienza di laboratorio non considerando il clock offset degli orologi a bordo dei satelliti della costellazione GPS.}}
		\label{fig:errclock}
	\end{figure}

	\vspace{-1cm}
	
	\begin{figure}[H]
		\centering
		\includegraphics[scale=0.31]{"Immagini workbook/Immagini esp3/err_leap_sec"}
		\caption{\textit{Errore commesso nella stima della posizione di partenza della terza esperienza di laboratorio non considerando l'esistenza dei leap seconds.}}
		\label{fig:errleap3}
	\end{figure}

	\vspace{-1cm}
		
	\begin{figure}[H]
		\centering
		\includegraphics[scale=0.31]{"Immagini workbook/Immagini esp3/err_sat_pos"}
		\caption{\textit{Errore commesso nella stima della posizione di partenza della terza esperienza di laboratorio considerando la posizione dei satelliti al momento di ricezione del segnale, e non al momento di emissione del segnale stesso.}}
		\label{fig:errsatpos3}
	\end{figure}

	\vspace{-1cm}
	
	\begin{figure}[H]
		\centering
		\includegraphics[scale=0.31]{"Immagini workbook/Immagini esp3/err_segnac"}
		\caption{\textit{Errore commesso nella stima della posizione di partenza della terza esperienza di laboratorio non considerando l'effetto Segnac, ovvero non tenendo in considerazione la rotazione della Terra.}}
		\label{fig:errsegnac3}
	\end{figure}

	\vspace{-1cm}
	
	\begin{figure}[H]
		\centering
		\includegraphics[scale=0.31]{"Immagini workbook/Immagini esp3/err_rel"}
		\caption{\textit{Errore commesso nella stima della posizione di partenza della terza esperienza di laboratorio non considerando l'effetto relativistico nella cadenza del tempo a bordo degli orologi dei satelliti.}}
		\label{fig:errrel3}
	\end{figure}

	\vspace{-1cm}
	
	\begin{figure}[H]
		\centering
		\includegraphics[scale=0.31]{"Immagini workbook/Immagini esp3/err_iono"}
		\caption{\textit{Errore commesso nella stima della posizione di partenza della terza esperienza di laboratorio non considerando l'effetto della ionosfera.}}
		\label{fig:erriono3}
	\end{figure}

	\vspace{-1cm}

	\begin{figure}[H]
		\centering
		\includegraphics[scale=0.31]{"Immagini workbook/Immagini esp3/err_tropo"}
		\caption{\textit{Errore commesso nella stima della posizione di partenza della terza esperienza di laboratorio non considerando l'effetto della troposfera.}}
		\label{fig:errtropo3}
	\end{figure}

	\clearpage
	
	\section{Appendice dell'Ottava Esperienza di Laboratorio}
	
	\vspace{-0.85cm}
	
	\begin{figure}[H]
		\centering
		\includegraphics[scale=0.30]{"Immagini workbook/Immagini esp8/err_clock_8"}
		\caption{\textit{Errore commesso nella stima delle posizioni acquisite nel caso dinamico dell'ottava esperienza di laboratorio non considerando il clock offset degli orologi a bordo dei satelliti della costellazione GPS.}}
		\label{fig:errclock8}
	\end{figure}
	
	\vspace{-0.7cm}
	
	\begin{figure}[H]
		\centering
		\includegraphics[scale=0.30]{"Immagini workbook/Immagini esp8/err_leap_8"}
		\caption{\textit{Errore commesso nella stima delle posizioni acquisite nel caso dinamico dell'ottava esperienza di laboratorio non considerando l'esistenza dei leap seconds.}}
		\label{fig:errleap8}
	\end{figure}
	
	\vspace{-0.7cm}
	
	\begin{figure}[H]
		\centering
		\includegraphics[scale=0.30]{"Immagini workbook/Immagini esp8/err_satpos_8"}
		\caption{\textit{Errore commesso nella stima delle posizioni acquisite nel caso dinamico dell'ottava esperienza di laboratorio considerando la posizione dei satelliti al momento di ricezione del segnale, e non al momento di emissione del segnale stesso.}}
		\label{fig:errsatpos8}
	\end{figure}
	
	\vspace{-1cm}
	
	\begin{figure}[H]
		\centering
		\includegraphics[scale=0.30]{"Immagini workbook/Immagini esp8/err_segnac_8"}
		\caption{\textit{Errore commesso nella stima delle posizioni acquisite nel caso dinamico dell'ottava esperienza di laboratorio non considerando l'effetto Segnac, ovvero non tenendo in considerazione la rotazione della Terra.}}
		\label{fig:errsegnac8}
	\end{figure}
	
	\vspace{-1cm}
	
	\begin{figure}[H]
		\centering
		\includegraphics[scale=0.30]{"Immagini workbook/Immagini esp8/err_rel_8"}
		\caption{\textit{Errore commesso nella stima delle posizioni acquisite nel caso dinamico dell'ottava esperienza di laboratorio non considerando l'effetto relativistico nella cadenza del tempo a bordo degli orologi dei satelliti.}}
		\label{fig:errrel8}
	\end{figure}
	
	\vspace{-1cm}
	
	\begin{figure}[H]
		\centering
		\includegraphics[scale=0.30]{"Immagini workbook/Immagini esp8/err_iono_8"}
		\caption{\textit{Errore commesso nella stima delle posizioni acquisite nel caso dinamico dell'ottava esperienza di laboratorio non considerando l'effetto della ionosfera.}}
		\label{fig:erriono8}
	\end{figure}
	
	\vspace{-1cm}
	
	\begin{figure}[H]
		\centering
		\includegraphics[scale=0.30]{"Immagini workbook/Immagini esp8/err_tropo_8"}
		\caption{\textit{Errore commesso nella stima delle posizioni acquisite nel caso dinamico dell'ottava esperienza di laboratorio non considerando l'effetto della troposfera.}}
		\label{fig:errtropo8}
	\end{figure}
	
\end{document}